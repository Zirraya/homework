\documentclass[bachelor, och, referat, times]{SCWorks}

% параметр - тип обучения - одно из значений:
%    spec     - специальность
%    bachelor - бакалавриат (по умолчанию)
%    master   - магистратура
% параметр - форма обучения - одно из значений:
%    och   - очное (по умолчанию)
%    zaoch - заочное
% параметр - тип работы - одно из значений:
%    referat    - реферат
%    coursework - курсовая работа (по умолчанию)
%    diploma    - дипломная работа
%    pract      - отчет по практике
%    pract      - отчет о научно-исследовательской работе
%    autoref    - автореферат выпускной работы
%    assignment - задание на выпускную квалификационную работу
%    review     - отзыв руководителя
%    critique   - рецензия на выпускную работу
% параметр - включение шрифта
%    times    - включение шрифта Times New Roman (если установлен)
%               по умолчанию выключен 



\usepackage[colorlinks=true]{hyperref}
\usepackage[T2A]{fontenc}
\usepackage[utf8]{inputenc}
\usepackage[english,russian]{babel}
\usepackage{graphicx}
\usepackage[sort,compress]{cite}
\usepackage{amsmath}
\usepackage{amssymb}
\usepackage{amsthm}
\usepackage{fancyvrb}
\usepackage{longtable}
\usepackage{array}
\usepackage{minted}
\usepackage{tempora}


\newcommand{\eqdef}{\stackrel {\rm def}{=}}
\newcommand{\No}{\textnumero}
\newtheorem{lem}{Лемма}
\setminted{style=bw,
	linenos=true,
	breaklines=true,
	numbersep=5pt,
	tabsize=2,
	fontsize=\small,
	bgcolor=white}
\setmintedinline{style=bw,
	bgcolor=white,
	fontsize=\normalsize
	}


%убирание преносов слов
\tolerance = 1
\emergencystretch = \maxdimen
\hbadness = 10000

\newcommand{\eqdef}{\stackrel {\rm def}{=}}

\newtheorem{lem}{Лемма}

\begin{document}
    
   % Кафедра (в родительном падеже)
    %\chair{математической кибернетики и компьютерных наук}
    \chair{информатики и программирования}
    % Тема работы
    \title{Анализ двоичного дерева поиска}
    
    % Курс
    \course{2}
    
    % Группа
    \group{211}
    
    % Факультет (в родительном падеже) (по умолчанию "факультета КНиИТ")
    \department{факультета компьютерных наук и информационных технологий}
    
    % Специальность/направление код - наименование
    
    \napravlenie{02.03.02 "--- Фундаментальная информатика и информационные технологии}
    
    % Для студентки. Для работы студента следующая команда не нужна.
    \studenttitle{студентки}
    
    % Фамилия, имя, отчество в родительном падеже
    \author{Никитенко Яны Валерьевны}
    
    % Заведующий кафедрой 
    %\chtitle{доцент, к.\,ф.-м.\,н.}
    %\chname{С.\,В.\,Миронов}
    % Научный руководитель (для реферата преподаватель проверяющий работу)
    %\satitle{доцент, к.\,ф.-м.\,н.} %должность, степень, звание
    %\saname{А.\,П.\,Грецова}
    % Руководитель ДПП ПП для цифровой кафедры (перекрывает заведующего кафедры)
    % \chpretitle{
    %     заведующий кафедрой математических основ информатики и олимпиадного\\
    %     программирования на базе МАОУ <<Ф"=Т лицей №1>>
    % }
    % \chtitle{г. Саратов, к.\,ф.-м.\,н., доцент}
    % \chname{Кондратова\, Ю.\,Н.}
    \date{2025}
    
    
    % Руководитель практики от организации (руководитель для цифровой кафедры)
    %\patitle{доцент, к.\,ф.-м.\,н.}
    %\paname{С.\,В.\,Миронов}
    
    % Руководитель НИР
    %\nirtitle{доцент, к.\,п.\,н.} % степень, звание
    %\nirname{В.\,А.\,Векслер}
    
    % Семестр (только для практики, для остальных типов работ не используется)
    %\term{2}
    
    % Наименование практики (только для практики, для остальных типов работ не
    % используется)
    %\practtype{учебная}
    
    % Продолжительность практики (количество недель) (только для практики, для
    % остальных типов работ не используется)
    %\duration{2}
    
    % Даты начала и окончания практики (только для практики, для остальных типов
    % работ не используется)
    %\practStart{01.07.2022}
    %\practFinish{13.01.2023}
    
    % Год выполнения отчета
   
    
    \maketitle

    
    
    % Включение нумерации рисунков, формул и таблиц по разделам
    % (по умолчанию - нумерация сквозная)
    % (допускается оба вида нумерации)
    %\secNumbering

    
    
   \tableofcontents
    
    % Раздел "Обозначения и сокращения". Может отсутствовать в работе
    %\abbreviations
    %\begin{description}
    %    \item $|A|$  "--- количество элементов в конечном множестве $A$;
    %    \item $\det B$  "--- определитель матрицы $B$;
    %    \item ИНС "--- Искусственная нейронная сеть;
    %    \item FANN "--- Feedforward Artifitial Neural Network
    %\end{description}
    
    % Раздел "Определения". Может отсутствовать в работе
    %\definitions
    
    % Раздел "Определения, обозначения и сокращения". Может отсутствовать в работе.
    % Если присутствует, то заменяет собой разделы "Обозначения и сокращения" и "Определения"
    %\defabbr
    

    \section{Текст программы}
    
    \begin{verbatim}
   
//
struct node {
	int key;
	node* left;
	node* right;
	node(int k) : key(k), left(nullptr), right(nullptr) {}
};

//

HANDLE outp = GetStdHandle(STD_OUTPUT_HANDLE);
CONSOLE_SCREEN_BUFFER_INFO csbInfo;
node* root = nullptr;

void max_height(node* x, short& max, short deepness = 1) 
{ // требует проверки на существование корня
	if (deepness > max) max = deepness;
	if (x->left) max_height(x->left, max, deepness + 1);
	if (x->right) max_height(x->right, max, deepness + 1);
}
bool isSizeOfConsoleCorrect(const short& width, const short& height) {
	GetConsoleScreenBufferInfo(outp, &csbInfo);
	COORD szOfConsole = csbInfo.dwSize;
	if (szOfConsole.X < width && szOfConsole.Y < height) 
    cout << "Please increase the height and width of the terminal. ";
	else if (szOfConsole.X < width) cout << "Please increase the width of the terminal. ";
	else if (szOfConsole.Y < height) cout << "Please increase the height of the terminal. ";
	if (szOfConsole.X < width || szOfConsole.Y < height) {
		cout << "Size of your terminal now: " << szOfConsole.X << ' ' << szOfConsole.Y
			<< ". Minimum required: " << width << ' ' << height << ".\n";
		return false;
	}
	
		return true;
}

void print_helper(node* x, const COORD pos, const short offset) {
	SetConsoleCursorPosition(outp, pos);
	cout << setw(offset + 1) << x->key;
	if (x->left) print_helper(x->left, { pos.X, short(pos.Y + 1) }, offset >> 1);
	if (x->right) print_helper(x->right, { short(pos.X + offset), short(pos.Y + 1) }, offset >> 1);
}

void print() {
	if (root) {
		short max = 1;
		max_height(root, max);
		short width = 1 << max + 1, max_w = 128; // вычисляем ширину вывода
		if (width > max_w) width = max_w;
		while (!isSizeOfConsoleCorrect(width, max)) system("pause");
		for (short i = 0; i < max; ++i) cout << '\n'; // резервируем место для вывода
		GetConsoleScreenBufferInfo(outp, &csbInfo); // получаем данные
		COORD endPos = csbInfo.dwCursorPosition;
		print_helper(root, { 0, short(endPos.Y - max) }, width >> 1);
		SetConsoleCursorPosition(outp, endPos);
		SetConsoleTextAttribute(outp, 7);
	}
}



// Добавить узел
void insert(node*& root, int key) {
    if (!root) {
        root = new node(key);
    }
    else if (key < root->key) {
        insert(root->left, key);
    }
    else {
        insert(root->right, key);
    }
}
//

// Найти 
node* findMin(node* root) {
    while (root && root->left) {
        root = root->left;
    }
    return root;
}
//

// Удалить узел
node* deleteNode(node* root, int key) {
    if (!root) return root;
    if (key < root->key) {
        root->left = deleteNode(root->left, key);
    }
    else if (key > root->key) {
        root->right = deleteNode(root->right, key);
    }
    else {
        if (!root->left) {
            node* temp = root->right;
            delete root;
            return temp;
        }
        else if (!root->right) {
            node* temp = root->left;
            delete root;
            return temp;
        }
        node* temp = findMin(root->right);
        root->key = temp->key;
        root->right = deleteNode(root->right, temp->key);
    }
    return root;
}
//

// Поиск
node* search(node* root, int key) {
    if (!root || root->key == key) return root;
    if (key < root->key) return search(root->left, key);
    return search(root->right, key);
}
//

// Обход в прямом порядке
void inorder(node* root) {
    if (root) {
        inorder(root->left);
        cout << root->key << " ";
        inorder(root->right);

    }
}
//

// Обход в симметричном порядке
void preorder(node* root) {
    if (root) {
        cout << root->key << " ";
        preorder(root->left);
        preorder(root->right);
    }
}
//

// Обход в обратном порядке
void postorder(node* root) {
    if (root) {
        postorder(root->left);
        postorder(root->right);
        cout << root->key << " ";
    }
}
//

// Меню
void menu() {
    cout << "1. Добавить узел\n";
    cout << "2. Удалить узел\n";
    cout << "3. Поиск узла\n";
    cout << "4. Обход в прямом порядке\n";
    cout << "5. Обход в симметричном порядке\n";
    cout << "6. Обход в обратном порядке\n";
    cout << "7. Вывести дерево\n";
    cout << "0. Выход\n";
}
//

int main() {
    setlocale(LC_ALL, "Russian");
    int choice, key;
    while (true) {
        menu();
        cout << "Выберите действие: ";
        cin >> choice;

        switch (choice) {
        case 1:
            while (true) {
                cout << "-1 - выход из цикла" << endl;
                cin >> key;
                if (key == -1) break; // Выход из цикла добавления
                insert(root, key);
            }
            break;

        case 2:
            cout << "ключ для удаления: ";
            cin >> key;
            root = deleteNode(root, key);
            break;
        case 3:
            cout << "ключ для поиска: ";
            cin >> key;
            if (search(root, key)) {
                cout << "узел с ключом " << key << " найден.\n";
            }
            else {
                cout << "узел с ключом " << key << " не найден.\n";
            }
            break;
        case 4:
            cout << "обход в прямом порядке: ";
            preorder(root);
            cout << endl;
            break;
        case 5:
            cout << "обход в симметричном порядке: ";
            inorder(root);
            cout << endl;
            break;
        case 6:
            cout << "обход в обратном порядке: ";
            postorder(root);
            cout << endl;
            break;
        case 7:
            print();
            break;
        case 0:
            return 0;
        default:
            cout << "\n";
        }
    }
    return 0;
}
 \end{verbatim}


 \section{Анализ случаев}



\textbf{ Лучший случай}

Идеально сбалансированное дерево, 
то есть высота одного поддерева отличается от высоты другого не более чем на 1. 
Высота такого дерева равна $log_2 N$, где N — количество элементов в дереве.
\vspace{5mm}

\textbf{Худший случай}

Дерево не сбалансированное и имеет только одно поддерево. 
Высота такого дерева будет равна N .
\vspace{5mm}

\textbf{Средний случай}

Любое другое дерево бинарного поиска. Его высота также составит
$log_2N$.


\section{Вставка в двоичном дереве}

\textbf{ Лучший случай}

В идеально сбалансированном дереве нам так же будет необходимо сделать 
лишь максимум $log_2N$ сравнений для поиска подходящего места для вставляемого узла, 
следовательно, временная сложность вставки в лучшем случае составит O(logN).
\vspace{5mm}

\textbf{Худший случай}

Нужно пройти от корня до последнего или самого глубокого листового узла, 
а максимальное количество шагов равно N (высота дерева). 
Таким образом, временная сложность составит O(N), 
так как поиск каждого узла один за другим до последнего листового узла займёт время O(N), 
а затем мы вставим элемент, что занимает константное время.
\vspace{5mm}

\textbf{Средний случай}

Проведя рассуждения, аналогичные тем, 
что были рассмотрены в среднем случае поиска элемента, получаем, 
что сложность операции вставки в среднем случае составит O(logN).



 \section{Удаление в двоичном дереве}

\textbf{ Лучший случай}

Снова максимальным значением количества проходов (сравнений) по дереву будет 
$log_2N$ — высота дерева. Копирование содержимого и его удаление требуют константного времени. 
Поэтому общая временная сложность составит O(logN).


\textbf{Худший случай}

Процесс удаления займёт O(N) времени, так как максимальное количество проходов 
(сравнений) по дереву равно N.


\textbf{Средний случай}

Проведя рассуждения, аналогичные тем, 
что были рассмотрены в среднем случае поиска элемента, 
получаем, что сложность операции удаления в среднем случае составит O(logN).

 

  \section{Поиск в двоичном дереве}


\textbf{Лучший случай}

Проходим через узлы один за другим. 
Если мы найдем элемент на втором уровне, 
то для этого мы сделаем 2 сравнения, если на третьем — 3 сравнения и так далее. 
Таким образом, на поиск ключа в дереве бинарного поиска мы затратим время, 
равное высоте дерева, то есть $log_2N$, поэтому временная сложность 
 поиска в лучшем случае составит O(logN).
 \vspace{5mm}


\textbf{Худший случай}

Нужно пройти от корня до самого глубокого узла, являющегося листом, 
и в этом случае высота дерева становится равной N, 
где N — количество элементов в дереве, и затрачиваемое время совпадает с высотой дерева. 
Поэтому временная сложность в худшем случае составит O(N).
\vspace{5mm}

\textbf{Средний случай}

Пусть S(N) — среднее значение общей длины внутреннего пути. 
Докажем, что временная сложность в этом случае составит O(logN).

Очевидно, что для дерева с одним узлом S(1) = 0. 
Любое бинарное дерево с N узлами содержит i элементов в левом поддереве, 
$0 \leq  i \leq  N - 1$, а в правом поддереве n - i - 1. Для фиксированного i получим:

 S(N) = (n - 1) + S(i) + S(n - i - 1), где  (n - 1)  - сумма дополнительных шагов к каждому узлу, 
 учитывая увеличение глубины всех узлов на 1; 
 S(i) - это суммарный внутренний путь в левом поддереве; 
 S(n - i - 1) - это суммарный внутренний путь в правом поддереве. 
 После суммирования этих повторений для $0 \leq  i \leq  N - 1$ получим:

$ S(n)=n(n-1)+\sum_{i=0}^{n-1} S(i) $


Следовательно, $S(N) \in  O(NlogN)$, и глубина узла $S(N) \in  O(logN)$.


  \section{Обходы дерева}

  Дерево бинарного поиска имеет 3 основных обхода: прямой, 
  обратный и симметричный. Их разница заключается в том, 
  в каком порядке мы обращаемся к элементам. 
  Каждый из них будет иметь временную сложность O(N), 
  так как процедура вызывается ровно два раза для каждого узла дерева.

  \section{Расход памяти}


  В двоичном дереве поиска каждый узел содержит значение и 
  указатели на левого и правого потомков. 
  Расход памяти в двоичном дереве поиска зависит от количества узлов и размера каждого узла.
Предположим, что каждый узел имеет фиксированный размер, состоящий из:

\begin{itemize}
\item Значения элемента (например, целочисленное значение)

\item Указателя на левого потомка (обычно 4 байта на 32-битной системе или 8 байт 
на 64-битной системе)

\item  Указателя на правого потомка (также 4 или 8 байт)
\end{itemize}

Следовательно, общий размер каждого узла составляет от 12 до 24 байтов в зависимости от архитектуры.
Память, затраченная на двоичное дерево поиска, зависит от количества узлов и их размера. 
Пусть n - количество узлов в дереве.
Тогда общий расход памяти для дерева будет составлять O(n), так как
каждый узел требует фиксированного количества памяти и 
количество узлов напрямую пропорционально объему занимаемой памяти.





    \appendix
     
\end{document}