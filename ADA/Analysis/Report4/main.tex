\documentclass[bachelor, och, referat, times]{SCWorks}

% параметр - тип обучения - одно из значений:
%    spec     - специальность
%    bachelor - бакалавриат (по умолчанию)
%    master   - магистратура
% параметр - форма обучения - одно из значений:
%    och   - очное (по умолчанию)
%    zaoch - заочное
% параметр - тип работы - одно из значений:
%    referat    - реферат
%    coursework - курсовая работа (по умолчанию)
%    diploma    - дипломная работа
%    pract      - отчет по практике
%    pract      - отчет о научно-исследовательской работе
%    autoref    - автореферат выпускной работы
%    assignment - задание на выпускную квалификационную работу
%    review     - отзыв руководителя
%    critique   - рецензия на выпускную работу
% параметр - включение шрифта
%    times    - включение шрифта Times New Roman (если установлен)
%               по умолчанию выключен 



\usepackage[colorlinks=true]{hyperref}
\usepackage[T2A]{fontenc}
\usepackage[utf8]{inputenc}
\usepackage[english,russian]{babel}
\usepackage{graphicx}
\usepackage[sort,compress]{cite}
\usepackage{amsmath}
\usepackage{amssymb}
\usepackage{amsthm}
\usepackage{fancyvrb}
\usepackage{longtable}
\usepackage{array}
\usepackage{minted}
\usepackage{tempora}


\newcommand{\eqdef}{\stackrel {\rm def}{=}}
\newcommand{\No}{\textnumero}
\newtheorem{lem}{Лемма}
\setminted{style=bw,
	linenos=true,
	breaklines=true,
	numbersep=5pt,
	tabsize=2,
	fontsize=\small,
	bgcolor=white}
\setmintedinline{style=bw,
	bgcolor=white,
	fontsize=\normalsize
	}


%убирание преносов слов
\tolerance = 1
\emergencystretch = \maxdimen
\hbadness = 10000

\newcommand{\eqdef}{\stackrel {\rm def}{=}}

\newtheorem{lem}{Лемма}

\begin{document}
    
   % Кафедра (в родительном падеже)
    %\chair{математической кибернетики и компьютерных наук}
    \chair{информатики и программирования}
    % Тема работы
    \title{Анализ красно-чёрного дерева}
    
    % Курс
    \course{2}
    
    % Группа
    \group{211}
    
    % Факультет (в родительном падеже) (по умолчанию "факультета КНиИТ")
    \department{факультета компьютерных наук и информационных технологий}
    
    % Специальность/направление код - наименование
    
    \napravlenie{02.03.02 "--- Фундаментальная информатика и информационные технологии}
    
    % Для студентки. Для работы студента следующая команда не нужна.
    \studenttitle{студентки}
    
    % Фамилия, имя, отчество в родительном падеже
    \author{Никитенко Яны Валерьевны}
    
    % Заведующий кафедрой 
    %\chtitle{доцент, к.\,ф.-м.\,н.}
    %\chname{С.\,В.\,Миронов}
    % Научный руководитель (для реферата преподаватель проверяющий работу)
    %\satitle{доцент, к.\,ф.-м.\,н.} %должность, степень, звание
    %\saname{А.\,П.\,Грецова}
    % Руководитель ДПП ПП для цифровой кафедры (перекрывает заведующего кафедры)
    % \chpretitle{
    %     заведующий кафедрой математических основ информатики и олимпиадного\\
    %     программирования на базе МАОУ <<Ф"=Т лицей №1>>
    % }
    % \chtitle{г. Саратов, к.\,ф.-м.\,н., доцент}
    % \chname{Кондратова\, Ю.\,Н.}
    \date{2025}
    
    
    % Руководитель практики от организации (руководитель для цифровой кафедры)
    %\patitle{доцент, к.\,ф.-м.\,н.}
    %\paname{С.\,В.\,Миронов}
    
    % Руководитель НИР
    %\nirtitle{доцент, к.\,п.\,н.} % степень, звание
    %\nirname{В.\,А.\,Векслер}
    
    % Семестр (только для практики, для остальных типов работ не используется)
    %\term{2}
    
    % Наименование практики (только для практики, для остальных типов работ не
    % используется)
    %\practtype{учебная}
    
    % Продолжительность практики (количество недель) (только для практики, для
    % остальных типов работ не используется)
    %\duration{2}
    
    % Даты начала и окончания практики (только для практики, для остальных типов
    % работ не используется)
    %\practStart{01.07.2022}
    %\practFinish{13.01.2023}
    
    % Год выполнения отчета
   
    
    \maketitle

    
    
    % Включение нумерации рисунков, формул и таблиц по разделам
    % (по умолчанию - нумерация сквозная)
    % (допускается оба вида нумерации)
    %\secNumbering

    
    
   \tableofcontents
    
    % Раздел "Обозначения и сокращения". Может отсутствовать в работе
    %\abbreviations
    %\begin{description}
    %    \item $|A|$  "--- количество элементов в конечном множестве $A$;
    %    \item $\det B$  "--- определитель матрицы $B$;
    %    \item ИНС "--- Искусственная нейронная сеть;
    %    \item FANN "--- Feedforward Artifitial Neural Network
    %\end{description}
    
    % Раздел "Определения". Может отсутствовать в работе
    %\definitions
    
    % Раздел "Определения, обозначения и сокращения". Может отсутствовать в работе.
    % Если присутствует, то заменяет собой разделы "Обозначения и сокращения" и "Определения"
    %\defabbr
    

    \section{Текст программы}
    
    \begin{verbatim}
   
#define RED true
#define BLACK false

enum RBTColor { Black, Red };

// Структура для дерева. Определена как шаблон функции
template<class KeyType>
struct RBTNode {
    KeyType key;
    RBTColor color;
    RBTNode<KeyType>* left;
    RBTNode<KeyType>* right;
    RBTNode<KeyType>* parent;
    RBTNode(KeyType k, RBTColor c, RBTNode* p, RBTNode* l, RBTNode* r) :
        key(k), color(c), parent(p), left(l), right(r) { };
};
//

// Класс для дерева. Шаблон. Тут почти список функций 
template<class T>
class RBTree {
public:
    RBTree();
    ~RBTree();

    void insert(T key);
    void remove(T key);
    RBTNode<T>* search(T key);
    void print();
    void preOrder();
    void inOrder();
    void postOrder();

private:
    void leftRotate(RBTNode<T>*& root, RBTNode<T>* x);
    void rightRotate(RBTNode<T>*& root, RBTNode<T>* y);
    void insert(RBTNode<T>*& root, RBTNode<T>* node);
    void InsertFixUp(RBTNode<T>*& root, RBTNode<T>* node);
    void destroy(RBTNode<T>*& node);
    void remove(RBTNode<T>*& root, RBTNode<T>* node);
    void removeFixUp(RBTNode<T>*& root, RBTNode<T>* node, 
    RBTNode<T>* parent);
    RBTNode<T>* search(RBTNode<T>* node, T key) const;
    void print(RBTNode<T>* node) const;
    void preOrder(RBTNode<T>* tree) const;
    void inOrder(RBTNode<T>* tree) const;
    void postOrder(RBTNode<T>* tree) const;

    RBTNode<T>* root;
    HANDLE outp = GetStdHandle(STD_OUTPUT_HANDLE);
    CONSOLE_SCREEN_BUFFER_INFO csbInfo;

    void max_height(RBTNode<T>* x, short& max, short deepness = 1) {
        if (deepness > max) max = deepness;
        if (x->left) max_height(x->left, max, deepness + 1);
        if (x->right) max_height(x->right, max, deepness + 1);
    }

    // Измененная функция print_helper
    void print_helper(RBTNode<T>* x, const COORD pos, const short offset) {
        SetConsoleTextAttribute(outp, x->color == RED ? 12 : 8);
        SetConsoleCursorPosition(outp, pos);
        cout << setw(offset + 1) << x->key;
        if (x->left) print_helper(x->left, { pos.X, short(pos.Y + 1) }, offset >> 1);
        if (x->right) print_helper(x->right, { short(pos.X + offset), short(pos.Y + 1) }, 
        offset >> 1);
    }

    bool isSizeOfConsoleCorrect(const short& width, const short& height) {
        GetConsoleScreenBufferInfo(outp, &csbInfo);
        COORD szOfConsole = csbInfo.dwSize;
        if (szOfConsole.X < width && szOfConsole.Y < height) 
        cout << "Please increase the height and width of the terminal. ";
        else if (szOfConsole.X < width) 
        cout << "Please increase the width of the terminal. ";
        else if (szOfConsole.Y < height) 
        cout << "Please increase the height of the terminal. ";
        if (szOfConsole.X < width || szOfConsole.Y < height) {
            cout << "Size of your terminal now: 
            " << szOfConsole.X << ' ' << szOfConsole.Y
                << ". Minimum required: " << width << ' ' << height << ".\n";
            return false;
        }
        return true;
    }
};
//

// Конструктор
template<class T>
RBTree<T>::RBTree() : root(nullptr) {
    root = nullptr;
}
//

// Деструктор
template<class T>
RBTree<T>::~RBTree() {
    destroy(root);
}
//

//
template<class T>
void RBTree<T>::leftRotate(RBTNode<T>*& root, RBTNode<T>* x) {
    RBTNode<T>* y = x->right;
    x->right = y->left;
    if (y->left != NULL)
        y->left->parent = x;
    y->parent = x->parent;
    if (x->parent == NULL)
        root = y;
    else {
        if (x == x->parent->left)
            x->parent->left = y;
        else
            x->parent->right = y;
    }
    y->left = x;
    x->parent = y;
};
//

//
template<class T>
void RBTree<T>::rightRotate(RBTNode<T>*& root, RBTNode<T>* y) {
    RBTNode<T>* x = y->left;
    y->left = x->right;
    if (x->right != NULL)
        x->right->parent = y;
    x->parent = y->parent;
    if (y->parent == NULL)
        root = x;
    else {
        if (y == y->parent->right)
            y->parent->right = x;
        else
            y->parent->left = x;
    }
    x->right = y;
    y->parent = x;
};
//

// Публичный метод для вставки
template<class T>
void RBTree<T>::insert(T key) {
    RBTNode<T>* z = new RBTNode<T>(key, Red, NULL, NULL, NULL);
    insert(root, z);
};
//

// Добавить узел. Основная функция
template<class T>
void RBTree<T>::insert(RBTNode<T>*& root, RBTNode<T>* node) {
    RBTNode<T>* x = root;
    RBTNode<T>* y = NULL;
    while (x != NULL) {
        y = x;
        if (node->key > x->key)
            x = x->right;
        else
            x = x->left;
    }
    node->parent = y;
    if (y != NULL) {
        if (node->key > y->key)
            y->right = node;
        else
            y->left = node;
    }
    else
        root = node;
    node->color = Red;
    InsertFixUp(root, node);
};
//

// Восстановление дерева после вставки
template<class T>
void RBTree<T>::InsertFixUp(RBTNode<T>*& root,
RBTNode<T>* node) {
    RBTNode<T>* parent;
    parent = node->parent;
    while (node != root && parent->color == Red) {
        RBTNode<T>* gparent = parent->parent;
        if (gparent->left == parent) {
            RBTNode<T>* uncle = gparent->right;
            if (uncle != NULL && uncle->color == Red) {
                parent->color = Black;
                uncle->color = Black;
                gparent->color = Red;
                node = gparent;
                parent = node->parent;
            }
            else {
                if (parent->right == node) {
                    leftRotate(root, parent);
                    swap(node, parent);
                }
                rightRotate(root, gparent);
                gparent->color = Red;
                parent->color = Black;
                break;
            }
        }
        else {
            RBTNode<T>* uncle = gparent->left;
            if (uncle != NULL && uncle->color == Red) {
                gparent->color = Red;
                parent->color = Black;
                uncle->color = Black;
                node = gparent;
                parent = node->parent;
            }
            else {
                if (parent->left == node) {
                    rightRotate(root, parent);
                    swap(parent, node);
                }
                leftRotate(root, gparent);
                parent->color = Black;
                gparent->color = Red;
                break;
            }
        }
    }
    root->color = Black;
}
//

// Функция для рекурсивного удаления указанного узла
template<class T>
void RBTree<T>::destroy(RBTNode<T>*& node) {
    if (node == NULL)
        return;
    destroy(node->left);
    destroy(node->right);
    delete node;
    node = nullptr;
}
//

// Публичный метод для следующей функции
template<class T>
void RBTree<T>::remove(T key) {
    RBTNode<T>* deletenode = search(root, key);
    if (deletenode != NULL)
        remove(root, deletenode);
}
//

// Удаление узла, и все все проблемы которые могут быть с этим связаны
template<class T>
void RBTree<T>::remove(RBTNode<T>*& root, RBTNode<T>* node) {
    RBTNode<T>* child, * parent;
    RBTColor color;
    if (node->left != NULL && node->right != NULL) {
        RBTNode<T>* replace = node;
        replace = node->right;
        while (replace->left != NULL) {
            replace = replace->left;
        }
        if (node->parent != NULL) {
            if (node->parent->left == node)
                node->parent->left = replace;
            else
                node->parent->right = replace;
        }
        else
            root = replace;
        child = replace->right;
        parent = replace->parent;
        color = replace->color;
        if (parent == node)
            parent = replace;
        else {
            if (child != NULL)
                child->parent = parent;
            parent->left = child;
            replace->right = node->right;
            node->right->parent = replace;
        }
        replace->parent = node->parent;
        replace->color = node->color;
        replace->left = node->left;
        node->left->parent = replace;
        if (color == Black)
            removeFixUp(root, child, parent);
        delete node;
        return;
    }
    if (node->left != NULL)
        child = node->left;
    else
        child = node->right;
    parent = node->parent;
    color = node->color;
    if (child) {
        child->parent = parent;
    }
    if (parent) {
        if (node == parent->left)
            parent->left = child;
        else
            parent->right = child;
    }
    else
        root = child;
    if (color == Black) {
        removeFixUp(root, child, parent);
    }
    delete node;
}
//

// Восстановление дерева после удаления узла
template<class T>
void RBTree<T>::removeFixUp(RBTNode<T>*& root, 
RBTNode<T>* node, RBTNode<T>* parent) {
    RBTNode<T>* othernode;
    while ((!node) || node->color == Black && node != root) {
        if (parent->left == node) {
            othernode = parent->right;
            if (othernode->color == Red) {
                othernode->color = Black;
                parent->color = Red;
                leftRotate(root, parent);
                othernode = parent->right;
            }
            else {
                if (!(othernode->right) || othernode->right->color == Black) {
                    othernode->left->color = Black;
                    othernode->color = Red;
                    rightRotate(root, othernode);
                    othernode = parent->right;
                }
                othernode->color = parent->color;
                parent->color = Black;
                othernode->right->color = Black;
                leftRotate(root, parent);
                node = root;
                break;
            }
        }
        else {
            othernode = parent->left;
            if (othernode->color == Red) {
                othernode->color = Black;
                parent->color = Red;
                rightRotate(root, parent);
                othernode = parent->left;
            }
            if ((!othernode->left || othernode->left->color == Black) && 
            (!othernode->right || othernode->right->color == Black)) {
                othernode->color = Red;
                node = parent;
                parent = node->parent;
            }
            else {
                if (!(othernode->left) || othernode->left->color == Black) {
                    othernode->right->color = Black;
                    othernode->color = Red;
                    leftRotate(root, othernode);
                    othernode = parent->left;
                }
                othernode->color = parent->color;
                parent->color = Black;
                othernode->left->color = Black;
                rightRotate(root, parent);
                node = root;
                break;
            }
        }
    }
    if (node)
        node->color = Black;
}
//

// Публчный метод для поиска
template<class T>
RBTNode<T>* RBTree<T>::search(T key) {
    return search(root, key);
}
//

// Функция для поиска уза
template<class T>
RBTNode<T>* RBTree<T>::search(RBTNode<T>* node, T key) const {
    if (node == NULL || node->key == key)
        return node;
    else
        if (key > node->key)
            return search(node->right, key);
        else
            return search(node->left, key);
}
//

// Функция для вывода
template<class T>
void RBTree<T>::print() {
    if (root == NULL)
        cout << "Пусто.\n";
    else {
        short max = 1;
        max_height(root, max);
        short width = 1 << max + 1, max_w = 128;
        if (width > max_w) width = max_w;
        while (!isSizeOfConsoleCorrect(width, max)) system("pause");
        for (short i = 0; i < max; ++i) cout << '\n';
        GetConsoleScreenBufferInfo(outp, &csbInfo);
        COORD endPos = csbInfo.dwCursorPosition;
        print_helper(root, { 0, short(endPos.Y - max) }, width >> 1);
        SetConsoleCursorPosition(outp, endPos);
        SetConsoleTextAttribute(outp, 7); // чтоб интерфейс не окрашивался
    }
}
//

// Обход
template<class T>
void RBTree<T>::preOrder() {
    if (root == NULL)
        cout << "Пусто.\n";
    else
        preOrder(root);
};
//

// Обход в прямом порядке
template<class T>
void RBTree<T>::preOrder(RBTNode<T>* tree) const {
    if (tree != NULL) {
        cout << tree->key << (tree->color == RED ? 12 : 8);
        preOrder(tree->left);
        preOrder(tree->right);
    }
}
//

//
template<class T>
void RBTree<T>::inOrder() {
    if (root == NULL)
        cout << "Пусто.\n";
    else
        inOrder(root);
};
//

// Обход в симметричном порядке
template<class T>
void RBTree<T>::inOrder(RBTNode<T>* tree) const {
    if (tree != NULL) {
        inOrder(tree->left);
        cout << tree->key << (tree->color == tree->color == RED ? 12 : 8);
        inOrder(tree->right);
    }
}
//

//
template<class T>
void RBTree<T>::postOrder() {
    if (root == NULL)
        cout << "Пусто.\n";
    else
        postOrder(root);
};
//

// Обход в обратном порядке
template<class T>
void RBTree<T>::postOrder(RBTNode<T>* tree) const {
    if (tree != NULL) {
        postOrder(tree->left);
        postOrder(tree->right);
        cout << tree->key << (tree->color == tree->color == RED ? 12 : 8);
    }
}
//


// Функция для вывода меню
void menu() {
    cout << "1. Добавить узлы\n";
    cout << "2. Удалить узел\n";
    cout << "3. Вывести дерево\n";
    cout << "4. Поиск узла\n";
    cout << "5. Обход в прямом порядке\n";
    cout << "6. Обход в симметричном порядке\n";
    cout << "7. Обход в обратном порядке\n";
    cout << "0. Выход\n";
}
//

int main()
{

    setlocale(LC_ALL, "Russian");
    RBTree<int> rbtree;
    int choice, key;
    while (true) {
        menu();
        cout << "Выберите действие: ";
        cin >> choice;

        switch (choice) {
        case 1: {
            cout << "(-1 - выход):\n";
            while (true) {
                cin >> key;
                if (key == -1) break;
                rbtree.insert(key);
                cout << key << " добавлен.\n";
            }
            break;
        }
        case 2: {
            cin >> key;
            rbtree.remove(key);
            cout << key << " удалён.\n";
            break;
        }
        case 3:
            rbtree.print();
            break;
        case 4:{
            cout << "ключ для поиска: ";
            cin >> key;
            if (rbtree.search(key)){
                cout << "узел с ключом " << key << " найден.\n";
            }
            else {
                cout << "узел с ключом " << key << " не найден.\n";
            }
            break;

         }
        case 5:
            cout << "Обход в прямом порядке: ";
            rbtree.preOrder();
            cout << endl;
            break;
        case 6:
            cout << "Обход в симметричном порядке: ";
            rbtree.inOrder();
            cout << endl;
            break;
        case 7:
            cout << "Обход в обратном порядке: ";
            rbtree.postOrder();
            cout << endl;
            break;
        case 0:
            return 0;
        default:
            cout << "Неверный выбор!\n";
        }
    }
    return 0;
    
}

 \end{verbatim}




\section{Операция вставки}

Операция вставки состоит из трёх этапов:
\begin{enumerate}
    \item Поиск места для вставки: $O(\log n)$
    \item Вставка нового узла (красного): $O(1)$
    \item Восстановление свойств дерева: $O(\log n)$
\end{enumerate}

\textbf{Сложность:}
\begin{itemize}
    \item \textbf{Лучший случай:} $O(\log n)$ — когда не требуется перекрашивание и повороты
    \item \textbf{Худший случай:} $O(\log n)$ — требуется до 2 поворотов и несколько перекрашиваний
    \item \textbf{Средний случай:} $O(\log n)$
\end{itemize}

\textbf{Анализ} 

В худшем случае может потребоваться поднятие от нового узла до корня с выполнением константного количества операций на каждом уровне, что даёт $O(\log n)$.


\section{Операция удаления}

Операция удаления состоит из:
\begin{enumerate}
    \item Поиска удаляемого узла: $O(\log n)$
    \item Собственно удаления: $O(1)$
    \item Восстановления свойств дерева: $O(\log n)$
\end{enumerate}

\textbf{Сложность:}
\begin{itemize}
    \item \textbf{Лучший случай:} $O(\log n)$
    \item \textbf{Худший случай:} $O(\log n)$ — требуется до 3 поворотов
    \item \textbf{Средний случай:} $O(\log n)$
\end{itemize}

\textbf{Анализ} 
В процессе восстановления свойств после удаления может потребоваться поднятие 
от удалённого узла до корня с выполнением константного количества операций на каждом у


\section{Операция поиска}

\textbf{Сложность:}
\begin{itemize}
    \item \textbf{Лучший случай:} $O(1)$ — если искомый элемент в корне
    \item \textbf{Худший случай:} $O(\log n)$
    \item \textbf{Средний случай:} $O(\log n)$
\end{itemize}

\textbf{Анализ} 

Так как высота дерева ограничена $2\log_2(n+1)$, поиск никогда не превышает $O(\log n)$.

  \section{Обходы дерева}
Все три типа обхода (прямой, симметричный, обратный) имеют сложность:
$$O(n)$$
где $n$ — количество узлов в дереве, так как каждый узел посещается ровно один раз.

  \section{Расход памяти}

Память, потребляемая красно-чёрным деревом:
\begin{itemize}
    \item Каждый узел содержит:
    \begin{itemize}
        \item Ключ (4 байта для int)
        \item Цвет (1 бит, обычно упаковывается в выравнивание)
        \item 3 указателя: на левого, правого потомка и родителя (24 байта на 64-битной системе)
    \end{itemize}
    \item Общий размер узла: примерно 28-32 байта
    \item Общая память для $n$ узлов: $O(n)$
\end{itemize}

Дополнительная память для рекурсивных операций: $O(\log n)$.

    \appendix
     
\end{document}