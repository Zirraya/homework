\documentclass[bachelor, och, referat, times]{SCWorks}

% параметр - тип обучения - одно из значений:
%    spec     - специальность
%    bachelor - бакалавриат (по умолчанию)
%    master   - магистратура
% параметр - форма обучения - одно из значений:
%    och   - очное (по умолчанию)
%    zaoch - заочное
% параметр - тип работы - одно из значений:
%    referat    - реферат
%    coursework - курсовая работа (по умолчанию)
%    diploma    - дипломная работа
%    pract      - отчет по практике
%    pract      - отчет о научно-исследовательской работе
%    autoref    - автореферат выпускной работы
%    assignment - задание на выпускную квалификационную работу
%    review     - отзыв руководителя
%    critique   - рецензия на выпускную работу
% параметр - включение шрифта
%    times    - включение шрифта Times New Roman (если установлен)
%               по умолчанию выключен 



\usepackage[colorlinks=false, pdfborder={0 0 0}]{hyperref}  % убирает и цвет, и рамки
\usepackage[T2A]{fontenc}
\usepackage[utf8]{inputenc}
\usepackage[english,russian]{babel}
\usepackage{graphicx}
\usepackage[sort,compress]{cite}
\usepackage{amsmath}
\usepackage{amssymb}
\usepackage{amsthm}
\usepackage{fancyvrb}
\usepackage{longtable}
\usepackage{array}
\usepackage{minted}
\usepackage{tempora}
\usepackage{listings}
\usepackage{verbatim}



\newcommand{\eqdef}{\stackrel {\rm def}{=}}
\newcommand{\No}{\textnumero}
\newtheorem{lem}{Лемма}
\setminted{style=bw,
	linenos=true,
	breaklines=true,
	numbersep=5pt,
	tabsize=2,
	fontsize=\small,
	bgcolor=white}
\setmintedinline{style=bw,
	bgcolor=white,
	fontsize=\normalsize
	}


%убирание преносов слов
\tolerance = 1
\emergencystretch = \maxdimen
\hbadness = 10000

\newcommand{\eqdef}{\stackrel {\rm def}{=}}

\newtheorem{lem}{Лемма}

\begin{document}
    
   % Кафедра (в родительном падеже)
    %\chair{математической кибернетики и компьютерных наук}
    \chair{информатики и программирования}
    % Тема работы
    \title{Анализ АВЛ дерева}
    
    % Курс
    \course{2}
    
    % Группа
    \group{211}
    
    % Факультет (в родительном падеже) (по умолчанию "факультета КНиИТ")
    \department{факультета компьютерных наук и информационных технологий}
    
    % Специальность/направление код - наименование
    
    \napravlenie{02.03.02 "--- Фундаментальная информатика и информационные технологии}
    
    % Для студентки. Для работы студента следующая команда не нужна.
    \studenttitle{студентки}
    
    % Фамилия, имя, отчество в родительном падеже
    \author{Никитенко Яны Валерьевны}
    
    % Заведующий кафедрой 
    %\chtitle{доцент, к.\,ф.-м.\,н.}
    %\chname{С.\,В.\,Миронов}
    % Научный руководитель (для реферата преподаватель проверяющий работу)
    %\satitle{доцент, к.\,ф.-м.\,н.} %должность, степень, звание
    %\saname{А.\,П.\,Грецова}
    % Руководитель ДПП ПП для цифровой кафедры (перекрывает заведующего кафедры)
    % \chpretitle{
    %     заведующий кафедрой математических основ информатики и олимпиадного\\
    %     программирования на базе МАОУ <<Ф"=Т лицей №1>>
    % }
    % \chtitle{г. Саратов, к.\,ф.-м.\,н., доцент}
    % \chname{Кондратова\, Ю.\,Н.}
    \date{2025}
    
    
    % Руководитель практики от организации (руководитель для цифровой кафедры)
    %\patitle{доцент, к.\,ф.-м.\,н.}
    %\paname{С.\,В.\,Миронов}
    
    % Руководитель НИР
    %\nirtitle{доцент, к.\,п.\,н.} % степень, звание
    %\nirname{В.\,А.\,Векслер}
    
    % Семестр (только для практики, для остальных типов работ не используется)
    %\term{2}
    
    % Наименование практики (только для практики, для остальных типов работ не
    % используется)
    %\practtype{учебная}
    
    % Продолжительность практики (количество недель) (только для практики, для
    % остальных типов работ не используется)
    %\duration{2}
    
    % Даты начала и окончания практики (только для практики, для остальных типов
    % работ не используется)
    %\practStart{01.07.2022}
    %\practFinish{13.01.2023}
    
    % Год выполнения отчета
   
    
    \maketitle

    
    
    % Включение нумерации рисунков, формул и таблиц по разделам
    % (по умолчанию - нумерация сквозная)
    % (допускается оба вида нумерации)
    %\secNumbering

    
    
   \tableofcontents
    
    % Раздел "Обозначения и сокращения". Может отсутствовать в работе
    %\abbreviations
    %\begin{description}
    %    \item $|A|$  "--- количество элементов в конечном множестве $A$;
    %    \item $\det B$  "--- определитель матрицы $B$;
    %    \item ИНС "--- Искусственная нейронная сеть;
    %    \item FANN "--- Feedforward Artifitial Neural Network
    %\end{description}
    
    % Раздел "Определения". Может отсутствовать в работе
    %\definitions
    
    % Раздел "Определения, обозначения и сокращения". Может отсутствовать в работе.
    % Если присутствует, то заменяет собой разделы "Обозначения и сокращения" и "Определения"
    %\defabbr
    

    \section{Текст программы}
    
    \begin{verbatim}


        // Структура дерева
struct node {
    int key;
    node* left;
    node* right;
    int height;
    node(int k) : key(k), left(nullptr), right(nullptr), height(1) {}
};
//

//
node* root = nullptr;
HANDLE outp = GetStdHandle(STD_OUTPUT_HANDLE);
CONSOLE_SCREEN_BUFFER_INFO csbInfo;
//

// Высота
int height(node* n) {
    return n ? n->height : 0;
}
//

// Вычисляет балансировачный фактор, нужно чтоб поддерживать дерево (и меня)
int balanceFactor(node* n) {
    return n ? height(n->left) - height(n->right) : 0;
}

// Обновляет высоту дерева, 
чтобы поддерживать высоту и корректность рабботы других операций
void updateHeight(node* n) {
    if (n) {
        n->height = 1 + max(height(n->left), height(n->right));
    }
}
//

// Вращение дерева, для балансирования дерева. Правое вращение балансирует, 
когда происходит вставка или удаления узла
// и дерево становиться не сблалонсированным
node* rotateRight(node* y) {
    node* x = y->left;
    node* T2 = x->right;

    x->right = y;
    y->left = T2;

    updateHeight(y);
    updateHeight(x);

    return x;
}
//


// Вращение дерева, для балансирования дерева. 
Левое вращение балансирует, когда происходит вставка или удаления узла
// и дерево становиться не сблалонсированным
node* rotateLeft(node* x) {
    node* y = x->right;
    node* T2 = y->left;

    y->left = x;
    x->right = T2;

    updateHeight(x);
    updateHeight(y);

    return y;
}
//

// Проверяет балансировочный фактор узла и
 выполняет необходимые вращения для исправления несбалансированности.
node* balance(node* n) {
    if (!n) return n;

    updateHeight(n);
    int bf = balanceFactor(n);


    if (bf > 1 && balanceFactor(n->left) >= 0)
        return rotateRight(n);


    if (bf > 1 && balanceFactor(n->left) < 0) {
        n->left = rotateLeft(n->left);
        return rotateRight(n);
    }


    if (bf < -1 && balanceFactor(n->right) <= 0)
        return rotateLeft(n);


    if (bf < -1 && balanceFactor(n->right) > 0) {
        n->right = rotateRight(n->right);
        return rotateLeft(n);
    }

    return n;
}
//

//
void max_height(node* x, short& max, short deepness = 1) {
    if (deepness > max) max = deepness;
    if (x->left) max_height(x->left, max, deepness + 1);
    if (x->right) max_height(x->right, max, deepness + 1);
}
//

//
bool isSizeOfConsoleCorrect(const short& width, const short& height) {
    GetConsoleScreenBufferInfo(outp, &csbInfo);
    COORD szOfConsole = csbInfo.dwSize;
    if (szOfConsole.X < width && szOfConsole.Y < height) 
    cout << "Please increase the height and width of the terminal. ";
    else if (szOfConsole.X < width) 
    cout << "Please increase the width of the terminal. ";
    else if (szOfConsole.Y < height) 
    cout << "Please increase the height of the terminal. ";
    if (szOfConsole.X < width || szOfConsole.Y < height) {
        cout << "Size of your terminal now: " << szOfConsole.X << ' ' << szOfConsole.Y
            << ". Minimum required: " << width << ' ' << height << ".\n";
        return false;
    }
    return true;
}
//

//
void print_helper(node* x, const COORD pos, const short offset) {
    SetConsoleCursorPosition(outp, pos);
    cout << setw(offset + 1) << x->key;
    if (x->left) print_helper(x->left, { pos.X, short(pos.Y + 1) }, offset >> 1);
    if (x->right) print_helper(x->right, { short(pos.X + offset), 
    short(pos.Y + 1) }, offset >> 1);
}
//

//
void print() {
    if (root == NULL) 
    {
        cout << "Пусто\n";
    }
    else
    {
        short max = 1;
        max_height(root, max);
        short width = 1 << max + 1, max_w = 128;
        if (width > max_w) width = max_w;
        while (!isSizeOfConsoleCorrect(width, max)) system("pause");
        for (short i = 0; i < max; ++i) cout << '\n';
        GetConsoleScreenBufferInfo(outp, &csbInfo);
        COORD endPos = csbInfo.dwCursorPosition;
        print_helper(root, { 0, short(endPos.Y - max) }, width >> 1);
        SetConsoleCursorPosition(outp, endPos);
    }
}
//

// Добавить узел
node* insert(node* root, int key) {
    if (!root) return new node(key);

    if (key < root->key)
        root->left = insert(root->left, key);
    else if (key > root->key)
        root->right = insert(root->right, key);
    else
        return root; // Дубликаты не допускаются

    return balance(root);
}
//

//
node* findMin(node* root) {
    while (root && root->left) {
        root = root->left;
    }
    return root;
}
//

// Удаление узла
node* deleteNode(node* root, int key) {
    if (!root) return root;

    if (key < root->key)
        root->left = deleteNode(root->left, key);
    else if (key > root->key)
        root->right = deleteNode(root->right, key);
    else {
        if (!root->left || !root->right) {
            node* temp = root->left ? root->left : root->right;
            if (!temp) {
                temp = root;
                root = nullptr;
            }
            else {
                *root = *temp;
            }
            delete temp;
        }
        else {
            node* temp = findMin(root->right);
            root->key = temp->key;
            root->right = deleteNode(root->right, temp->key);
        }
    }

    if (!root) return root;

    return balance(root);
}
//

// Поиск узла
node* search(node* root, int key) {
    if (!root || root->key == key) return root;
    if (key < root->key) return search(root->left, key);
    return search(root->right, key);
}
//

// Обход в симметричном порядке
void inorder(node* root) {
    if (root) {
        inorder(root->left);
        cout << root->key << " ";
        inorder(root->right);
    }
}
//

// Обход в прямом порядке
void preorder(node* root) {

        cout << root->key << " ";
        preorder(root->left);
        preorder(root->right);
}
//

// Обход в обратном порядке
void postorder(node* root) {
    if (root) {
        postorder(root->left);
        postorder(root->right);
        cout << root->key << " ";
    }
}
//

// Меню
void menu() {
    cout << "1. Добавить узлы\n";
    cout << "2. Удалить узел\n";
    cout << "3. Вывести дерево\n";
    cout << "4. Поиск узла\n";
    cout << "5. Обход в прямом порядке\n";
    cout << "6. Обход в симметричном порядке\n";
    cout << "7. Обход в обратном порядке\n";
    cout << "0. Выход\n";
}
//

int main() {
    setlocale(LC_ALL, "Russian");
    int choice, key;
    while (true) {
        menu();
        cout << "Выберите действие: ";
        cin >> choice;

        switch (choice) {
        case 1:
            cout << "-1 - выход" << endl;
            while (true) {
                cin >> key;
                if (key == -1) break;
                root = insert(root, key);
                cout << key << " добавлен.\n";
            }
            break;

        case 2:
            cout << "ключ для удаления: ";
            cin >> key;
            root = deleteNode(root, key);
            cout << key << " удалён.\n";
            break;
        case 3:
            print();
            cout << endl;
            break;
        case 4:
            cout << "ключ для поиска: ";
            cin >> key;
            if (search(root, key)) {
                cout << "узел с ключом " << key << " найден.\n";
            }
            else {
                cout << "узел с ключом " << key << " не найден.\n";
            }
            break;
        case 5:
            cout << "обход в прямом порядке: ";
            preorder(root);
            cout << endl;
            break;
        case 6:
            cout << "обход в симметричном порядке: ";
            inorder(root);
            cout << endl;
            break;
        case 7:
            cout << "обход в обратном порядке: ";
            postorder(root);
            cout << endl;
            break;
        case 0:
            return 0;
        default:
            cout << "\n";
        }
    }
    return 0;
}
   

 \end{verbatim}


 \section{Вставка в АВЛ дереве}

 В худшем случае вставка требует времени O(logn), где n-количество элементов в дереве.
Это  происходит из-за необходимости сбалансировать дерево после каждой вставки, 
что занимает время, пропорциональное высоте дерева.
Следовательно, время вставки для n элементов будет O(nlogn).

 \section{Удаление в АВЛ дереве}

Как и вставка, удаление также требует времени O(logn) в худшем случае. 
После удаления элемента дерево также требуется сбалансировать.
Таким образом, время удаления для n элементов также будет O(nlogn).

 \section{Поиск в АВЛ дереве}

Время поиска в АВЛ-дереве также O(logn) в худшем случае.
Поиск выполняется по пути от корня до листа в дереве, 
при этом каждый раз отбрасывается половина оставшихся узлов.


 \section{Обход в АВЛ дереве}

Префиксный (preorder), постфиксный (postorder) и инфиксный (inorder) обходы занимают 
O(n) времени, так как каждый узел дерева должен быть посещен ровно один раз.
Таким образом, общая временная сложность операций вставки, удаления, 
поиска и обходов в АВЛ-дереве составляет O(nlogn) для n элементов.


  \section{Расход памяти}

  \begin{enumerate}
    
\item Узел дерева

Для каждого узла дерева выделяется фиксированное количество памяти, состоящее из:

inf: O(1) памяти

left и right: указатели на другие узлы дерева, каждый из которых занимает O(1) памяти

height: переменная типа unsignedchar, занимающая O(1) памяти

Таким образом, общее количество памяти, выделенное под каждый узел дерева, составляет O(1).


\item Локальные переменные и дополнительные данные

Локальные переменные, используемые в функциях, 
также требуют ограниченное количество памяти, 
которое не зависит от размера входных данных или высоты дерева.
Поэтому расход памяти на локальные переменные можно считать O(1).


\item Входные данные

Расход памяти на входные данные (значения, которые вставляются в дерево) 
также не зависит от размера дерева и может быть считан O(n), где n – количество элементов.


\item Стек вызовов


Во время выполнения рекурсивных операций используется стек вызовов, 
чтобы хранить информацию о текущем состоянии выполнения каждой рекурсивной функции. 
Глубина стека вызовов зависит от высоты дерева и количества рекурсивных вызовов, что может быть O(logn) в худшем случае для операций вставки,удаления и поиска.
Таким образом, общий асимптотический анализ сложности расхода памяти составляет O(n+logn),
 где n - количество элементов в дереве.

Вращение происходит за время O(1). 
Временная сложность всех функций равна O(logN), 
потому что дерево всегда сбалансированное.

  \end{enumerate}

 


    \appendix
     
\end{document}