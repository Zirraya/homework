\documentclass{article}
\usepackage{graphicx} % Required for inserting images
\usepackage{amsmath}
\usepackage[english, russian] {babel}
\usepackage[utf8]{inputenc}
\usepackage[T2A]{fontenc}
\usepackage{minted}
\usepackage{float}
\usepackage{amssymb}

\title{АСА лекция}
\author{silvia.lesnaia }
\date{February 2025}

\begin{document}

\maketitle

\text{11.02.25}

\section{Введение в теорию алгоритмов}

\section{Примеры интуитивного понятия алгоритма}

Алгоритм - точный набор инструкция, описывающих порядок действий исполнителя для достижения результата решения задачи за конечное время

Алгоритм - это понятные и точные предписания исполнителю совершить конченное число шагов, направленных на решение поставленной задачи

Алгоритм - это конеченый набор правил, который определяет последовательность операций


\susbsection{Основные свойства алгоритмов}

Дискретность

Детерминированность

Понятность

Завершаеомсть

Массовость

Однозначность результата 



\subsection{Основные задачи теории алгоритмов}

формализация понятия «алгоритм» и  исследование формальных алгоритмических систем;

формальное доказательство алгоритмической неразрешимости ряда задач;

классификация задач, определение и исследование сложностных классов;

асимптотический анализ сложности алгоритмов;

исследование и анализ рекурсивных алгоритмов;

получение явных функций трудоемкости в целях сравнительного анализа алгоритмов;

разработка критериев сравнительной оценки качества алгоритмов.

\subsection{Схема определения понятия «алгоритм»:}

Понятие данных

Память

Элементарный шаг

Детерминированность

Результативность

\subsection{Основные типы алгоритмических моделей}
Алгоритм как некое детерминированное устройство - абстрактные машины. 
Машина Тьюринга и машина Поста.

Алгоритм как процедура вычисления некой числовой функции. Рекурсивные функции Черча.

Алгоритм как последовательность преобразований цепочек в каком-либо алфавите.
(Комбинаторные операции над словами). Нормальные алгоритмы Маркова.

\section{Машина Поста}

Тезис Поста - “Всякий алгоритм представим в форме машины Поста”.

Алгоритм (по Посту) — программа для машины Поста, приводящая к решению поставленной задачи.

Если задача имеет алгоритмическое решение, то она представима в форме команд для машины Поста.


\begin{figure}
    \centering
    \includegraphics[width=1\linewidth]{Снимок экрана 2025-02-25 090545.png}
\end{figure}


\subsection{Варианты окончания выполнения программы на машине Поста}

останов по команде "стоп". 
Такой останов называется результативным и указывает на корректность алгоритма;

останов при выполнении недопустимой команды. 
Случаи, когда указатель должен записать метку там, где она уже есть, 
или стереть метку там, где ее нет;

машина не останавливается никогда. Уход в бесконечность, зацикливание.


\subsection{Примеры}

Пример: покажем, как можно воспользоваться командой условного перехода 
для организации циклического процесса. 
Пусть на ленте имеется запись из нескольких меток подряд, 
и головка находится над самой крайней меткой справа. 
Требуется перевести головку влево до первой пустой позиции.

			1\leftarrow 2

			2 ? 3; 1

			3 !


 Пример: увеличить число 3 на единицу (изменить значение в памяти с 3 на 4). 
 Допустим, точно известно, что каретка стоит где-то слева от меток и обозревает пустую ячейку. 
 Тогда программа увеличения числа на единицу может выглядеть так:

 1 -> 2

 2 ? 1;3

 3 <- 4

 4 V 5

 5 !
 


Пример: на ленте машины Поста расположен массив из n меток. 
Составить программу, действуя по которой машина выяснит, делится ли число n на 3. 
Если да, то после массива через одну пустую ячейку поставить метку.

1 \rightarrow 2

2 ? 3;4

3 !

4 \rightarrow 5

5 ? 3;6

6 \rightarrow 7

7 & 8;1

8 \rightarrow 9

9 V 3


Пример: зацикливание. 

1 → 2

2 ← 1


6 вариант 
\begin{figure}
    \centering
    \includegraphics[width=1\linewidth]{photo_5285451810783490009_y.jpg}

Ответ: 01010110
    
\end{figure}

\vspace{10mm}

\textbf{18.02.25}

\section{Машина Тюрингита}

\subsection{Формальное описание машины Тюрингита}

\subsection{Способы задания МТ}

    Граф переходов

    \subsection{Конфигурация МТ}

    Совокупность состояний ленты, указаний на ленте

    Протоколы - 

    \subsection{Приведений конфигураций к стандартному виду}

    \subsection{Определение вычислимости по Тьюрингу}

    \begin{tabular}{||c||c||c||c||c||}
\hline 1 & 2 & 3 & 4 & 5 \\
\hline \text{(1)} & L_1^4 & \text{(2)} & \{120, 011, 112\} & \text{(3)} \\
\hline \text{(4)} & L_2^2 & \text{(5)} & \{1\} & \\
\hline
\end{tabular}


\vspace{10mm}

\textbf{25.02.25}

\section{Приницип суперпозиции}

\section{Оператор примитивной рекурисии}


окончание перовй презентации тезис Черча

начало второй презентации

\section{Алгоритм Маркова}


\end{document}