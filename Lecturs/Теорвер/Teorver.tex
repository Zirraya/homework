\documentclass{article}
\usepackage{graphicx} % Required for inserting images
\usepackage{amsmath}
\usepackage[english, russian] {babel}
\usepackage[utf8]{inputenc}
\usepackage[T2A]{fontenc}
\usepackage{minted}
\usepackage{float}
\usepackage{amssymb}

\title{Теоривер лекция}
\author{silvia.lesnaia }
\date{February 2025}

\begin{document}

\maketitle

\text{11.02.25}

Тут нужн заполнить лекцию из тетради


\textbf{25.02.25}


Продолжение



Стах. экперимент: правильная монта побрасывается до тех пор, пока не выпадет орел.


пупупуп пустота

надо заполнить



Сумма геометрической прогрессии: $S= \frac{b_1}{1-q}$

Проверим:
$\sum_{k = 1}^{\infty} P_k = \sum_{k = 1}^{\infty} (\frac{1}{2})^k  
= \frac{1/2}{1-1/2} = 1$ 


Воспроизведем: $P_k = \frac{1}{2^k} k = 1,2$

Востановим вер-ть, того, что будет проиведено четное количесвто бросков:

$A = \left\{(0,0),(p,p,p,0)...\right\}  $

$P(A) = \sum_{k = 1}^{\infty} P_k I(A) = \frac{1}{2^2} + \frac{1}{2^4} + \frac{1}{2^6}
= \frac{1/4}{1-1/4} = \frac{4}{3}$



\section{Геометрическая вероятность пространство}

$(\Omega, \mathcal{F}, \mathcal{P} $



\subsection{Свойста вероятности}


1 $\mathcal{P} (\bar{A}) = 1 - \mathcal{P} (A)$


Док-во: Предствавим $\Omega = A\bigsqcup \bar{A}$
По аксиоме P1 $P(\Omega) = 1$
аксома P3 $P(\Omega) = P(A\bigsqcup \bar{A}) = P(A) + P(\bar{A}) \Rightarrow P(A) + P(\bar{A}) = 1$
$\Rightarrow P(\bar{A}) = 1 - P(A)$

Следствие: $P(\varnothing ) = 1 - P(\Omega) = 1-1=0$

\vspace{5mm}

2 Пусть $A \subseteq B $,

Тогда $P(A) \leqslant P(B), P(B/A) = P(B) - P(A)$

Док-во: Предствавим $B=A \sqcup (B \setminus  A)$

По аксиоме P3

$P(B) = P(A \sqcup (S) \setminus A ) = P(A)+P(B \setminus A) \eqslantgtr P(A)$

Т.к по аксиоме P2 $P(\bullet ) \geq 0 \Rightarrow P(B) \geq P(A)$

\vspace{5mm}

3 Теорема сложения вероятности

Пусть  $A,B \in \mathcal{F}$

Тогда $P(A\cup B) = P(A) + P(B) - P(A \cap  B)$

Док-во: Заменим $C = A \cup  B$ в виде суммы несове

Тогда $P(A\cup B) $

Следствие:

\vspace{5mm}

4 Свойство непрерывности вероятностной меры



\vspace{5mm}


\section{Условная вероятность}




Опр:




\subsection{Свойства услонвной вероятности}

1 




\textbf{Замечание} Таким образом услованя вероятнсть является вреосятностной мерой 
и удовлетворяет аксиомам Кормагорага,



\textbf{Теорема} умножение вероятностей


на след лекцию заголов

\section{Независимость событий}





\textbf{04.03.25}


\textbf{25.03.25}

\end{document}