\documentclass[bachelor, och, pract, times]{SCWorks}
% параметр - тип обучения - одно из значений:
%    spec     - специальность
%    bachelor - бакалавриат (по умолчанию)
%    master   - магистратура
% параметр - форма обучения - одно из значений:
%    och   - очное (по умолчанию)
%    zaoch - заочное
% параметр - тип работы - одно из значений:
%    referat    - реферат
%    coursework - курсовая работа (по умолчанию)
%    diploma    - дипломная работа
%    pract      - отчет по практике
%    pract      - отчет о научно-исследовательской работе
%    autoref    - автореферат выпускной работы
%    assignment - задание на выпускную квалификационную работу
%    review     - отзыв руководителя
%    critique   - рецензия на выпускную работу
% параметр - включение шрифта
%    times    - включение шрифта Times New Roman (если установлен)
%               по умолчанию выключен 
%убирание преносов слов
\tolerance = 1
\emergencystretch = \maxdimen
\hbadness = 10000

\usepackage{listings}

\usepackage[colorlinks=true]{hyperref}
\usepackage[T2A]{fontenc}
\usepackage[utf8]{inputenc}
\usepackage[english,russian]{babel}
\usepackage{graphicx}
\usepackage[sort,compress]{cite}
\usepackage{amsmath}
\usepackage{amssymb}
\usepackage{amsthm}
\usepackage{fancyvrb}
\usepackage{longtable}
\usepackage{array}
\usepackage{minted}
\usepackage{tempora}


\newcommand{\eqdef}{\stackrel {\rm def}{=}}
\newcommand{\No}{\textnumero}
\newtheorem{lem}{Лемма}
\setminted{style=bw,
	linenos=true,
	breaklines=true,
	numbersep=5pt,
	tabsize=2,
	fontsize=\small,
	bgcolor=white}
\setmintedinline{style=bw,
	bgcolor=white,
	fontsize=\normalsize
	}



\begin{document}
    
   % Кафедра (в родительном падеже)
    %\chair{математической кибернетики и компьютерных наук}
    \chair{информатики и программирования}
    % Тема работы
    \title{Проектная практика}
    
    % Курс
    \course{3}
    
    % Группа
    \group{311}
    
    % Факультет (в родительном падеже) (по умолчанию "факультета КНиИТ")
    \department{факультета компьютерных наук и информационных технологий}
    
    % Специальность/направление код - наименование
    
    \napravlenie{02.03.02 "--- Фундаментальная информатика и информационные технологии}
    
    % Для студентки. Для работы студента следующая команда не нужна.
    \studenttitle{студентки}
    
    % Фамилия, имя, отчество в родительном падеже
    \author{Никитенко Яны Валерьевны}

    % Заведующий кафедрой
\chtitle{доцент, к. ф.-м. н.} % степень, звание
\chname{С.\,В.\,Миронов}

%Научный руководитель (для реферата преподаватель проверяющий работу)
\satitle{доцент, к. ф.-м. н.}
\saname{А.\,С.\,Иванова}

% Руководитель практики от организации (только для практики,
% для остальных типов работ не используется)
\patitle{доцент, к. ф.-м. н.} 
\paname{А.\,С.\,Иванова}

% Семестр (только для практики, для остальных
% типов работ не используется)
\term{1}

% Наименование практики (только для практики, для остальных
% типов работ не используется)
\practtype{учебная}

% Семестр (только для практики, для остальных
% типов работ не используетс
    
    % Заведующий кафедрой 
    %\chtitle{доцент, к.\,ф.-м.\,н.}
    %\chname{С.\,В.\,Миронов}
    % Научный руководитель (для реферата преподаватель проверяющий работу)
    %\satitle{доцент, к.\,ф.-м.\,н.} %должность, степень, звание
    %\saname{А.\,П.\,Грецова}
    % Руководитель ДПП ПП для цифровой кафедры (перекрывает заведующего кафедры)
    % \chpretitle{
    %     заведующий кафедрой математических основ информатики и олимпиадного\\
    %     программирования на базе МАОУ <<Ф"=Т лицей №1>>
    % }
    % \chtitle{г. Саратов, к.\,ф.-м.\,н., доцент}
    % \chname{Кондратова\, Ю.\,Н.}
    \date{2025}
    
    
    % Руководитель практики от организации (руководитель для цифровой кафедры)
    %\patitle{доцент, к.\,ф.-м.\,н.}
    %\paname{С.\,В.\,Миронов}
    
    % Руководитель НИР
    %\nirtitle{доцент, к.\,п.\,н.} % степень, звание
    %\nirname{В.\,А.\,Векслер}
    
    % Семестр (только для практики, для остальных типов работ не используется)
    %\term{2}
    
    % Наименование практики (только для практики, для остальных типов работ не
    % используется)
    %\practtype{учебная}
    
    % Продолжительность практики (количество недель) (только для практики, для
    % остальных типов работ не используется)
    \duration{2}
    
    % Даты начала и окончания практики (только для практики, для остальных типов
    % работ не используется)
    \practStart{01.09.2025}
    \practFinish{30.12.2025 }
    
    % Год выполнения отчета
   
    
    \maketitle

    
    % Включение нумерации рисунков, формул и таблиц по разделам
    % (по умолчанию - нумерация сквозная)
    % (допускается оба вида нумерации)
    %\secNumbering

    
    
    \tableofcontents
    
    % Раздел "Обозначения и сокращения". Может отсутствовать в работе
    %\abbreviations
    %\begin{description}
    %    \item $|A|$  "--- количество элементов в конечном множестве $A$;
    %    \item $\det B$  "--- определитель матрицы $B$;
    %    \item ИНС "--- Искусственная нейронная сеть;
    %    \item FANN "--- Feedforward Artifitial Neural Network
    %\end{description}
    
    % Раздел "Определения". Может отсутствовать в работе
    %\definitions
    
    % Раздел "Определения, обозначения и сокращения". Может отсутствовать в работе.
    % Если присутствует, то заменяет собой разделы "Обозначения и сокращения" и "Определения"
    %\defabbr
    
    
    % Раздел "Введение"
 \intro   


Данный проект представляет собой одностраничное веб-приложение для управления списком дел "Сервис To-do", 
разработанное в рамках практического задания по фронтенд-разработке. 
Пользователь сможет просматривать, добавлять, удалять, копировать и редактировать задачи. 
А все данные будут сохраняться в локальное хранилище, 
чтобы они не терялись при перезагрузке страницы.

\section{Отображение текущих задач}   
При загрузке страницы пользователь должен видеть список текущих задач. Для этого 
реализовано две функции:
\begin{enumerate}
    \item Функция loadInitialTasks  
    
    Она отвечает за получение списка задач. служит для загрузки начального списка задач 
    из локального хранилища браузера (localStorage) или использования дефолтного массива items, 
    если сохранённых задач нет.

\begin{verbatim}
function loadInitialTasks() {
    const savedTasks = localStorage.getItem('items');
    return savedTasks ? JSON.parse(savedTasks) : [...items];
}
\end{verbatim}
Если savedTasks существует (не null/undefined), парсит JSON-строку в массив и возвращает его.

Если сохранённых задач нет, возвращает копию исходного массива items (через spread-оператор [...items])

\item Функция createTaskElement 

Она создает и возвращает готовый DOM-элемент задачи для списка дел, 
полностью настроенный со всеми необходимыми обработчиками событий.

\begin{verbatim}
    const template = document.getElementById("to-do__item-template");
    const taskElement=template.content.querySelector(".to-do__item").cloneNode(true);
\end{verbatim}


\vspace{5mm}

 Клонирование шаблона
\begin{itemize}
    \item Находит HTML-шаблон с id "to-do\_\_item-template".
    \item Клонирует элемент с классом .to-do\_\_item из содержимого шаблона.
    \item cloneNode(true) - глубокое клонирование (со всеми дочерними элементами).
\end{itemize}


\vspace{5mm}

Поиск внутренних элементов


\begin{verbatim}
 const taskTextElement = taskElement.querySelector(".to-do__item-text");
  const deleteButton = taskElement.querySelector(".to-do__item-button_type_delete");
  const duplicateButton = taskElement.querySelector(".to-do__item-button_type_duplicate");
  const editButton = taskElement.querySelector(".to-do__item-button_type_edit");
\end{verbatim}

Находит все необходимые элементы внутри клонированного шаблона:
\begin{itemize}
    \item Элемент для текста задачи
    \item Кнопку удаления
    \item Кнопку дублирования
    \item Кнопку редактирования
\end{itemize}

\vspace{5mm}

 Установка текста задачи

taskTextElement.textContent = taskText;

Устанавливает переданный текст задачи в соответствующий элемент

\vspace{5mm}

Назначение обработчиков событий на кнопки
\begin{verbatim}
deleteButton.addEventListener('click', () => {
  removeTask(taskElement);
});
\end{verbatim}
При клике на кнопку удаления вызывает функцию removeTask()

Дублирование задачи:
\begin{verbatim}
duplicateButton.addEventListener('click', () => {
  duplicateTask(taskTextElement);
});
\end{verbatim}

Редактирование задачи:
\begin{verbatim}
editButton.addEventListener('click', () => {
  enableTaskEditing(taskTextElement);
});
\end{verbatim}

\vspace{5mm}

Обработчики для редактирования текста

Завершение редактирования (потеря фокуса):
\begin{verbatim}
taskTextElement.addEventListener('blur', () => {
  finishTaskEditing(taskTextElement);
});
\end{verbatim}

Завершение редактирования (клавиша Enter):
\begin{verbatim}
taskTextElement.addEventListener('keydown', (event) => {
  if (event.key === 'Enter') {
    event.preventDefault();
    taskTextElement.blur();
  }
});
\end{verbatim}

\vspace{5mm}

 Возврат готового элемента

 return taskElement;

\end{enumerate}

   
\section{Создание и сохранение задач}

Установлен слушатель на событие submit формы и написан обработчик. 
Элемент формы уже доступен через переменную formElement.

Внутри обработчика:
\begin{itemize}
    \item Отключена перезагрузка страницы при отправке формы.
    \item Получение текста задачи из поля ввода. Элемент поля ввода доступен через переменную inputElement.
    \item Создание готовой разметки элемента задачи с помощью функции addNewTask() 
    и добавление её в начало контейнера .to-do\_\_list с помощью метода prepend. 
    Контейнер доступен через переменную listElement.
    \item Очищение поля ввода.
\end{itemize}

Функция getAllTasks - собирает список задач из текущей разметки и возвращает его в виде массива строк.
\begin{verbatim}
 function getAllTasks() {
  const taskElements = listElement.querySelectorAll('.to-do__item-text');
  const currentTasks = [];
  
  taskElements.forEach(element => {
    currentTasks.push(element.textContent);
  });
  
  return currentTasks;
}   
\end{verbatim}


Функция updateStoredTasks - сохраняет в локальное хранилище переданный в параметре массив строк задач
\begin{verbatim}
function updateStoredTasks() {
  const currentTasks = getAllTasks();
  localStorage.setItem('items', JSON.stringify(currentTasks));
}    
\end{verbatim}

\section{Удаление задачи}
Добавлен обработчик события click. 
Кнопка удаления задачи уже доступна через переменную deleteButton.

Внутри обработчика:
\begin{itemize}
    \item Удаление текущего элемента задачи, с помощью метода remove
    \item Создание переменной items. Результат выполнения функции getAllTasks
    \item Сохранение в локальное хранилище с помощью вызова функции updateStoredTasks
\end{itemize}


\section{Копирование задачи}

К кнопке с классом .to-do\_\_item-button\_type\_duplicate добавлен обработчик событий click.

Внутри обработчика:
\begin{itemize}
    \item Получение текста текущей задачи из элемента taskTextElement
    \item Создание нового элемента задачи с помощью функции createTaskElement, передавая текущий текст
    \item Добавление новой задачи в начало списка с помощью метода prepend
    \item Обновление данных в локальном хранилище с помощью функции updateStoredTasks
\end{itemize}

Пример кода функции duplicateTask:
\begin{verbatim}
function duplicateTask(taskTextElement) {
  const currentText = taskTextElement.textContent;
  const newTaskElement = createTaskElement(currentText);
  listElement.prepend(newTaskElement);
  updateStoredTasks();
}
\end{verbatim}

\section{Редактирование задачи}

Для редактирования задачи реализованы две функции:
\begin{itemize}
    \item Функция enableTaskEditing - включает режим редактирования для элемента задачи
    \item Функция finishTaskEditing - завершает редактирование и сохраняет изменения
\end{itemize}

Пример кода функций:
\begin{verbatim}
function enableTaskEditing(taskTextElement) {
  taskTextElement.setAttribute('contenteditable', 'true');
  taskTextElement.focus();
}

function finishTaskEditing(taskTextElement) {
  taskTextElement.setAttribute('contenteditable', 'false');
  updateStoredTasks();
}
\end{verbatim}



\conclusion 

В ходе выполнения проектной работы было разработано одностраничное веб-приложение "Сервис To-do" 
с полным набором функций для управления задачами. Приложение позволяет:

\begin{itemize}
    \item Просматривать список текущих задач
    \item Добавлять новые задачи через форму ввода
    \item Удалять существующие задачи
    \item Дублировать задачи для быстрого создания похожих записей
    \item Редактировать текст задач непосредственно в интерфейсе
    \item Автоматически сохранять все изменения в локальное хранилище браузера
\end{itemize}

Были успешно реализованы все основные функции управления задачами, 
а также обеспечено сохранение данных между сессиями. 
Приложение имеет интуитивно понятный интерфейс и отвечает современным требованиям к веб-приложениям.


    
\end{document}