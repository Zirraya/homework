\documentclass[bachelor, och, referat, times]{SCWorks}

% параметр - тип обучения - одно из значений:
%    spec     - специальность
%    bachelor - бакалавриат (по умолчанию)
%    master   - магистратура
% параметр - форма обучения - одно из значений:
%    och   - очное (по умолчанию)
%    zaoch - заочное
% параметр - тип работы - одно из значений:
%    referat    - реферат
%    coursework - курсовая работа (по умолчанию)
%    diploma    - дипломная работа
%    pract      - отчет по практике
%    pract      - отчет о научно-исследовательской работе
%    autoref    - автореферат выпускной работы
%    assignment - задание на выпускную квалификационную работу
%    review     - отзыв руководителя
%    critique   - рецензия на выпускную работу
% параметр - включение шрифта
%    times    - включение шрифта Times New Roman (если установлен)
%               по умолчанию выключен 



\usepackage[colorlinks=true]{hyperref}
\usepackage[T2A]{fontenc}
\usepackage[utf8]{inputenc}
\usepackage[english,russian]{babel}
\usepackage{graphicx}
\usepackage[sort,compress]{cite}
\usepackage{amsmath}
\usepackage{amssymb}
\usepackage{amsthm}
\usepackage{fancyvrb}
\usepackage{longtable}
\usepackage{array}
\usepackage{minted}
\usepackage{tempora}


\newcommand{\eqdef}{\stackrel {\rm def}{=}}
\newcommand{\No}{\textnumero}
\newtheorem{lem}{Лемма}
\setminted{style=bw,
	linenos=true,
	breaklines=true,
	numbersep=5pt,
	tabsize=2,
	fontsize=\small,
	bgcolor=white}
\setmintedinline{style=bw,
	bgcolor=white,
	fontsize=\normalsize
	}


%убирание преносов слов
\tolerance = 1
\emergencystretch = \maxdimen
\hbadness = 10000

\newcommand{\eqdef}{\stackrel {\rm def}{=}}

\newtheorem{lem}{Лемма}

\begin{document}
    
   % Кафедра (в родительном падеже)
    %\chair{математической кибернетики и компьютерных наук}
    \chair{математической физики и вычислительной математики}
    % Тема работы
    \title{Отчет по практической подготовке}
    
    % Курс
    \course{3}
    
    % Группа
    \group{311}
    
    % Факультет (в родительном падеже) (по умолчанию "факультета КНиИТ")
    \department{факультета компьютерных наук и информационных технологий}
    
    % Специальность/направление код - наименование
    
    \napravlenie{02.03.02 "--- Фундаментальная информатика и информационные технологии}
    
    % Для студентки. Для работы студента следующая команда не нужна.
    \studenttitle{студентки}
    
    % Фамилия, имя, отчество в родительном падеже
    \author{Никитенко Яны Валерьевны}
    
    % Заведующий кафедрой 
    %\chtitle{доцент, к.\,ф.-м.\,н.}
    %\chname{С.\,В.\,Миронов}
    % Научный руководитель (для реферата преподаватель проверяющий работу)
    %\satitle{доцент, к.\,ф.-м.\,н.} %должность, степень, звание
    %\saname{А.\,П.\,Грецова}
    % Руководитель ДПП ПП для цифровой кафедры (перекрывает заведующего кафедры)
    % \chpretitle{
    %     заведующий кафедрой математических основ информатики и олимпиадного\\
    %     программирования на базе МАОУ <<Ф"=Т лицей №1>>
    % }
    % \chtitle{г. Саратов, к.\,ф.-м.\,н., доцент}
    % \chname{Кондратова\, Ю.\,Н.}
    \date{2025}
    
    
    % Руководитель практики от организации (руководитель для цифровой кафедры)
    %\patitle{доцент, к.\,ф.-м.\,н.}
    %\paname{С.\,В.\,Миронов}
    
    % Руководитель НИР
    %\nirtitle{доцент, к.\,п.\,н.} % степень, звание
    %\nirname{В.\,А.\,Векслер}
    
    % Семестр (только для практики, для остальных типов работ не используется)
    %\term{2}
    
    % Наименование практики (только для практики, для остальных типов работ не
    % используется)
    %\practtype{учебная}
    
    % Продолжительность практики (количество недель) (только для практики, для
    % остальных типов работ не используется)
    %\duration{2}
    
    % Даты начала и окончания практики (только для практики, для остальных типов
    % работ не используется)
    %\practStart{01.07.2022}
    %\practFinish{13.01.2023}
    
    % Год выполнения отчета
   
    
    \maketitle

    
    
    % Включение нумерации рисунков, формул и таблиц по разделам
    % (по умолчанию - нумерация сквозная)
    % (допускается оба вида нумерации)
    %\secNumbering

    
        
    % Раздел "Обозначения и сокращения". Может отсутствовать в работе
    %\abbreviations
    %\begin{description}
    %    \item $|A|$  "--- количество элементов в конечном множестве $A$;
    %    \item $\det B$  "--- определитель матрицы $B$;
    %    \item ИНС "--- Искусственная нейронная сеть;
    %    \item FANN "--- Feedforward Artifitial Neural Network
    %\end{description}
    
    % Раздел "Определения". Может отсутствовать в работе
    %\definitions
    
    % Раздел "Определения, обозначения и сокращения". Может отсутствовать в работе.
    % Если присутствует, то заменяет собой разделы "Обозначения и сокращения" и "Определения"
    %\defabbr
    

   
  

\textbf{Раздел 1}

1. По данным интерполяции построить интерполяционный многочлен в общем
виде, в форме Лагранжа и в форме Ньютона.


Исходные данные

X(k)   1   3   5   7

F(k)   0   7   2   18


Текст программы

\begin{verbatim}
#include <iostream>
#include <vector>
#include <iomanip>
#include <algorithm>
#include <cmath>
#include <string>
#include <locale>
using namespace std;
// Функция для ввода данных с клавиатуры
void inputData(vector<double>& x_data, vector<double>& f_data) {
    int n;
    cout << "Введите количество точек: ";
    cin >> n;
    x_data.resize(n);
    f_data.resize(n);
    cout << "\nВведите значения X:" << endl;
    for (int i = 0; i < n; i++) {
        cout << "X[" << i << "] = ";
        cin >> x_data[i];
    }
    cout << "\nВведите значения F(X):" << endl;
    for (int i = 0; i < n; i++) {
        cout << "F(" << x_data[i] << ") = ";
        cin >> f_data[i];
    }
}
// Функция для вывода введенных данных
void printInputData(const vector<double>& x_data, 
const vector<double>& f_data) {
    cout << "\nВведенные данные:" << endl;
    cout << setw(10) << "X" << setw(10) << "F(X)" << endl;
    cout << string(20, '-') << endl;
    for (int i = 0; i < x_data.size(); i++) {
        cout << setw(10) << x_data[i] << setw(10) << f_data[i] << endl;
    }
    cout << endl;
}
// Функция для решения системы линейных уравнений методом Гаусса
vector<double> solveLinearSystem(vector<vector<double>> A, 
vector<double> b) {
    int n = A.size();
    // Прямой ход метода Гаусса
    for (int i = 0; i < n; i++) {
        // Поиск главного элемента
        int maxRow = i;
        for (int k = i + 1; k < n; k++) {
            if (abs(A[k][i]) > abs(A[maxRow][i])) {
                maxRow = k;
            }
        }
        // Перестановка строк
        swap(A[i], A[maxRow]);
        swap(b[i], b[maxRow]);
        // Исключение
        for (int k = i + 1; k < n; k++) {
            double factor = A[k][i] / A[i][i];
            for (int j = i; j < n; j++) {
                A[k][j] -= factor * A[i][j];
            }
            b[k] -= factor * b[i];
        }
    }
    // Обратный ход
    vector<double> x(n);
    for (int i = n - 1; i >= 0; i--) {
        x[i] = b[i];
        for (int j = i + 1; j < n; j++) {
            x[i] -= A[i][j] * x[j];
        }
        x[i] /= A[i][i];
    }
    return x;
}
// Функция для вычисления значения полинома
double polynomial(double x, const vector<double>& coeffs) {
    double result = 0.0;
    for (int i = 0; i < coeffs.size(); i++) {
        result += coeffs[i] * pow(x, i);
    }
    return result;
}
// Функция для представления полинома в виде строки
string polynomialToString(const vector<double>& coeffs) {
    string result;
    int degree = coeffs.size() - 1;

    for (int i = 0; i < coeffs.size(); i++) {
        if (coeffs[i] != 0 || degree == 0) {
            if (!result.empty() && coeffs[i] >= 0) {
                result += " + ";
            }
            else if (!result.empty() && coeffs[i] < 0) {
                result += " - ";
            }
            string term;
            if (i == 0) {
                term = to_string(coeffs[i]);
            }
            else {
                double absCoef = abs(coeffs[i]);
                if (absCoef != 1.0) {
                    term = to_string(absCoef) + " * ";
                }
                term += "x";
                if (i > 1) {
                    term += "^" + to_string(i);
                }
            }
            // Удаляем лишние нули после запятой
            size_t pos = term.find('.');
            if (pos != string::npos) {
                size_t lastNonZero = term.find_last_not_of('0');
                if (lastNonZero == pos) {
                    term = term.substr(0, pos);
                }
                else {
                    term = term.substr(0, lastNonZero + 1);
                }
            }
            result += term;
        }
        degree--;
    }
    return result;
}
// Функция для вычисления коэффициентов многочлена Ньютона
vector<double> newtonCoefficients(const vector<double>& x, 
const vector<double>& y) {
    int n = x.size();
    vector<double> coeffs = y;
    for (int j = 1; j < n; j++) {
        for (int i = n - 1; i >= j; i--) {
            coeffs[i] = (coeffs[i] - coeffs[i - 1]) / (x[i] - x[i - j]);
        }
    }
    return coeffs;
}
// Функция для вычисления значения многочлена Ньютона
double newtonPolynomial(double x, const vector<double>& coeffs, 
const vector<double>& x_data) {
    int n = coeffs.size() - 1;
    double p = coeffs[n];

    for (int k = 1; k <= n; k++) {
        p = coeffs[n - k] + (x - x_data[n - k]) * p;
    }
    return p;
}
// Функция для вывода вектора
void printVector(const string& name, const vector<double>& vec) {
    cout << name << ": ";
    for (double val : vec) {
        cout << val << " ";
    }
    cout << endl;
}
// Функция для вывода матрицы
void printMatrix(const string& name, 
const vector<vector<double>>& matrix) {
    cout << name << ":" << endl;
    for (const auto& row : matrix) {
        for (double val : row) {
            cout << setw(8) << val << " ";
        }
        cout << endl;
    }
}
int main() {
     setlocale(LC_ALL, "Russian");   
    vector<double> x_data, f_data;
    // Меню выбора способа ввода данных
    int choice;
    cout << "Выберите способ ввода данных:" << endl;
    cout << "1 - Ввести данные с клавиатуры" << endl;
    cout << "2 - Использовать пример данных (X: 81, 21, 2, 3; F: 0, 7, 8, 28)" << endl;
    cout << "Ваш выбор: ";
    cin >> choice;
    if (choice == 1) {
        inputData(x_data, f_data);
    }
    else {
        // Пример данных по умолчанию
        x_data = { 81, 21, 2, 3 };
        f_data = { 0, 7, 8, 28 };
    }
    printInputData(x_data, f_data);
    // Проверка на достаточное количество точек
    if (x_data.size() < 2) {
        cout << "Ошибка: необходимо как минимум 
        2 точки для интерполяции!" << endl;
        return 1;
    }
    // Построение матрицы Вандермонда
    int n = x_data.size();
    vector<vector<double>> A(n, vector<double>(n));
    for (int i = 0; i < n; i++) {
        for (int j = 0; j < n; j++) {
            A[i][j] = pow(x_data[i], j);
        }
    }
    printMatrix("Матрица", A);
    cout << endl;
    // Решение системы уравнений для нахождения коэффициентов
    vector<double> coefficients = solveLinearSystem(A, f_data);
    cout << "Найденные коэффициенты:" << endl;
    for (int i = 0; i < coefficients.size(); i++) {
        cout << "a" << i << " = " << coefficients[i] << endl;
    }
    cout << endl;
    // Вывод уравнения многочлена
    cout << "Уравнение интерполяционного многочлена:" << endl;
    cout << polynomialToString(coefficients) << endl << endl;
    // Вычисление коэффициентов для многочлена Ньютона
    vector<double> newton_coeffs = newtonCoefficients(x_data, f_data);
    cout << "Коэффициенты многочлена Ньютона:" << endl;
    for (int i = 0; i < newton_coeffs.size(); i++) {
        cout << "b" << i << " = " << newton_coeffs[i] << endl;
    }
    cout << endl;
    // Расчет значений в промежуточных точках
    vector<double> x_half;
    for (int i = 0; i < x_data.size() - 1; i++) {
        x_half.push_back((x_data[i] + x_data[i + 1]) / 2.0);
    }
    // Объединение и сортировка всех точек
    vector<double> x_combined = x_data;
    x_combined.insert(x_combined.end(), x_half.begin(), x_half.end());
    sort(x_combined.begin(), x_combined.end());
    // Вывод итоговой таблицы
    cout << "Итоговая таблица:" << endl;
    cout << setw(10) << "X" << setw(15) << "F (исходные)"
        << setw(15) << "F (Лагранж)" << setw(15) << "F (Ньютон)" << endl;
    cout << string(55, '-') << endl;
    for (double x : x_combined) {
        // Проверяем, является ли x исходной точкой
        bool isOriginal = find(x_data.begin(), x_data.end(), x) != x_data.end();
        double original_f = 0.0;
        if (isOriginal) {
            auto it = find(x_data.begin(), x_data.end(), x);
            int index = distance(x_data.begin(), it);
            original_f = f_data[index];
        }
        double lagrange_f = polynomial(x, coefficients);
        double newton_f = newtonPolynomial(x, newton_coeffs, x_data);

        cout << fixed << setprecision(4);
        cout << setw(10) << x;
        if (isOriginal) {
            cout << setw(15) << original_f;
        }
        else {
            cout << setw(15) << "-";
        }
        cout << setw(15) << lagrange_f << setw(15) << newton_f << endl;
    }
    return 0;
}
\end{verbatim}

\vspace{20mm}

Вывод программы

Матрица:
\[
\begin{pmatrix}
3 & 9 & 27 \\
5 & 25 & 125 \\
7 & 49 & 343
\end{pmatrix}
\]
    Найденные коэффициенты:

$a_0 = -18.3125, \quad $
$a_1 = 25.3125, \quad $
$a_2 = -7.6875, \quad $
$a_3 = 0.6875$

Уравнение интерполяционного многочлена:

$P(x) = -18.312500 + 25.312500 \cdot x - 7.687500 \cdot x^2 + 0.687500 \cdot x^3$

Коэффициенты многочлена Ньютона:

$b_0 = 0, \quad 
b_1 = 3.5, \quad 
b_2 = -1.5, \quad 
b_3 = 0.6875$

Итоговая таблица значений:
\[
\begin{array}{|c|c|c|c|}
\hline
X & F\text{ (исходные)} & F\text{ (Лагранж)} & F\text{ (Ньютон)} \\
\hline
1.0000 & 0.0000 & 0.0000 & 0.0000 \\
2.0000 & -- & 7.0625 & 7.0625 \\
3.0000 & 7.0000 & 7.0000 & 7.0000 \\
4.0000 & -- & 3.9375 & 3.9375 \\
5.0000 & 2.0000 & 2.0000 & 2.0000 \\
6.0000 & -- & 5.3125 & 5.3125 \\
7.0000 & 18.0000 & 18.0000 & 18.0000 \\
\hline
\end{array}
\]

\newpage  % Новая страница

2. Построить кусочно-непрерывную
склейку кубических сплайнов.

Исходные данные

Узлы интерполяции (x и f(x)):

Узел 1 - x: 11

Узел 1 - f(x): 2

Узел 2 - x: 7

Узел 2 - f(x): 88

Узел 3 - x: 14

Узел 3 - f(x): 60

Узел 4 - x: 10

Узел 4 - f(x): 20

Узел 5 - x: 1

Узел 5 - f(x): 50

Коэффициенты кубического многочлена:

a3: 2

a2: 3

a1: 7

a0: 25

Текст программы

\begin{verbatim}
    
#include <iostream>
#include <vector>
#include <iomanip>
#include <algorithm>
#include <cmath>
#include <string>
#include <locale>
using namespace std;
// Структура для хранения коэффициентов полинома
struct PolynomialCoeffs {
    double a3, a2, a1, a0;
};
// Структура для хранения коэффициентов сплайна на отрезке
struct SplineSegment {
    double a, b, c, d;
    double x_left;
};
// Функция для решения СЛАУ методом прогонки (трехдиагональная матрица)
vector<double> solveTridiagonal(const vector<vector<double>>& A, 
const vector<double>& b) {
    int n = b.size();
    vector<double> alpha(n, 0), beta(n, 0), x(n, 0);
    // Прямой ход
    alpha[0] = -A[0][1] / A[0][0];
    beta[0] = b[0] / A[0][0];
    for (int i = 1; i < n - 1; i++) {
        double denominator = A[i][i] + A[i][i - 1] * alpha[i - 1];
        alpha[i] = -A[i][i + 1] / denominator;
        beta[i] = (b[i] - A[i][i - 1] * beta[i - 1]) / denominator;
    }
    // Обратный ход
    x[n - 1] = (b[n - 1] - A[n - 1][n - 2] * beta[n - 2]) /
        (A[n - 1][n - 1] + A[n - 1][n - 2] * alpha[n - 2]);
    for (int i = n - 2; i >= 0; i--) {
        x[i] = alpha[i] * x[i + 1] + beta[i];
    }
    return x;
}
// Вычисление значения полинома в точке x
double evaluatePolynomial(double x, 
const PolynomialCoeffs& coeffs) {
    return coeffs.a3 * x * x * x + coeffs.a2 * x * x + coeffs.a1 * x + coeffs.a0;
}
int main() {
    setlocale(LC_ALL, "Russian");
    vector<double> x_data, f_data;
    int n;
    // Ввод данных с клавиатуры
    cout << "Введите количество узлов интерполяции: ";
    cin >> n;
    cout << "Введите узлы интерполяции (x и f(x)):" << endl;
    for (int i = 0; i < n; i++) {
        double x, fx;
        cout << "Узел " << i + 1 << " - x: ";
        cin >> x;
        cout << "Узел " << i + 1 << " - f(x): ";
        cin >> fx;
        x_data.push_back(x);
        f_data.push_back(fx);
    }
    // Сортируем узлы по x
    vector<pair<double, double>> points;
    for (int i = 0; i < n; i++) {
        points.push_back({ x_data[i], f_data[i] });
    }
    sort(points.begin(), points.end());
    for (int i = 0; i < n; i++) {
        x_data[i] = points[i].first;
        f_data[i] = points[i].second;
    }
    // Ввод коэффициентов кубического полинома
    PolynomialCoeffs poly_coeffs;
    cout << "\nВведите коэффициенты кубического многочлена:" << endl;
    cout << "a3: ";
    cin >> poly_coeffs.a3;
    cout << "a2: ";
    cin >> poly_coeffs.a2;
    cout << "a1: ";
    cin >> poly_coeffs.a1;
    cout << "a0: ";
    cin >> poly_coeffs.a0;
    // Вычисление промежуточных точек
    vector<double> x_half;
    for (int i = 0; i < x_data.size() - 1; i++) {
        x_half.push_back((x_data[i] + x_data[i + 1]) / 2.0);
    }
    // Объединение всех точек
    vector<double> x_combined = x_data;
    x_combined.insert(x_combined.end(), x_half.begin(), x_half.end());
    sort(x_combined.begin(), x_combined.end());
    // Вывод таблицы
    cout << fixed << setprecision(3);
    cout << "\nТаблица с кубическим сплайном:" << endl;
    cout << "X\tF (исходные)\tF (Сплайн)" << endl;
    for (double x : x_combined) {
        // Проверяем, является ли точка исходной
        bool is_original = false;
        double original_value = 0.0;
        for (int i = 0; i < x_data.size(); i++) {
            if (abs(x - x_data[i]) < 1e-6) {
                is_original = true;
                original_value = f_data[i];
                break;
            }
        }
        double spline_value = evaluatePolynomial(x, poly_coeffs);
        if (is_original) {
            cout << x << "\t" << original_value 
            << "\t\t" << spline_value << endl;
        }
        else {
            cout << x << "\t\t\t" << spline_value << endl;
        }
    }
    // Вывод коэффициентов полинома
    cout << "\nКоэффициенты кубического многочлена:" << endl;
    cout << "a3 = " << fixed << setprecision(6) << poly_coeffs.a3 << endl;
    cout << "a2 = " << fixed << setprecision(6) << poly_coeffs.a2 << endl;
    cout << "a1 = " << fixed << setprecision(6) << poly_coeffs.a1 << endl;
    cout << "a0 = " << fixed << setprecision(6) << poly_coeffs.a0 << endl;
    // Построение кубического сплайна
    int m = x_data.size() - 1;
    vector<double> h(m);
    for (int i = 0; i < m; i++) {
        h[i] = x_data[i + 1] - x_data[i];
    }
    // Построение матрицы СЛАУ
    vector<vector<double>> A(m + 1, vector<double>(m + 1, 0.0));
    vector<double> b(m + 1, 0.0);
    // Граничные условия
    A[0][0] = 1.0;
    A[m][m] = 1.0;
    // Заполнение внутренних строк
    for (int i = 1; i < m; i++) {
        A[i][i - 1] = h[i - 1];
        A[i][i] = 2.0 * (h[i - 1] + h[i]);
        A[i][i + 1] = h[i];
        b[i] = 3.0 * ((f_data[i + 1] - f_data[i]) / h[i] -
            (f_data[i] - f_data[i - 1]) / h[i - 1]);
    }
    // Вывод матрицы СЛАУ
    cout << "Матрица СЛАУ для сплайна:" << endl;
    cout << fixed << setprecision(1);
    for (int i = 0; i <= m; i++) {
        cout << "[";
        for (int j = 0; j <= m; j++) {
            cout << A[i][j];
            if (j < m) cout << " ";
        }
        cout << "]" << endl;
    }
    // Вывод вектора правой части
    cout << "\nВектор правой части СЛАУ для сплайна:" << endl;
    for (int i = 0; i <= m; i++) {
        cout << " " << b[i];
        if (i < m) cout << endl;
    }
    // Решение СЛАУ для коэффициентов c
    vector<double> c = solveTridiagonal(A, b);
    return 0;
}
\end{verbatim}


Вывод программы

\[
\begin{array}{|c|c|c|}
\hline
X & F\text{ (исходные)} & F\text{ (Сплайн)} \\
\hline
1.000 & 50.000 & 37.000 \\
4.000 & & 229.000 \\
7.000 & 88.000 & 907.000 \\
8.500 & & 1529.500 \\
10.000 & 20.000 & 2395.000 \\
10.500 & & 2744.500 \\
11.000 & 2.000 & 3127.000 \\
12.500 & & 4487.500 \\
14.000 & 60.000 & 6199.000 \\
\hline
\end{array}
\]

Коэффициенты кубического многочлена:

$a_3 = 2.000000,\quad a_2 = 3.000000,\quad a_1 = 7.000000,\quad a_0 = 25.000000$


Матрица СЛАУ:
\[
\begin{bmatrix}
1.0 & 0.0 & 0.0 & 0.0 & 0.0 \\
6.0 & 18.0 & 3.0 & 0.0 & 0.0 \\
0.0 & 3.0 & 8.0 & 1.0 & 0.0 \\
0.0 & 0.0 & 1.0 & 8.0 & 3.0 \\
0.0 & 0.0 & 0.0 & 0.0 & 1.0
\end{bmatrix}
\]

Вектор правой части СЛАУ:
\[
\begin{bmatrix}
 0.0 \\
 -87.0 \\
 14.0 \\
 112.0 \\
 0.0
\end{bmatrix}
\]

\newpage  % Новая страница

\textbf{Раздел 2}

1. Решить систему линейных алгебраических уравнений методом Гаусса.

Исходные данные

Матрица A :

\[
\begin{bmatrix}
16 & 1.6 & 1.6 & 1.6 & 1.6 \\
1.7 & 17 & 1.7 & 1.7 & 1.7 \\
1.8 & 1.8 & 18 & 1.8 & 1.8 \\
1.9 & 1.9 & 1.9 & 19 & 1.9 \\
2 & 2 & 2 & 2 & 20
\end{bmatrix}
\]

Вектор b: 16 17 18 19 20 

Текст программы

\begin{verbatim}

#include <iostream>
#include <vector>
#include <iomanip>
#include <cmath>
#include <algorithm>
using namespace std;
// Функция для вывода матрицы
void printMatrix(const vector<vector<double>>& A, const vector<double>& b) {
    int n = A.size();
    cout << "\nМатрица системы A|b:\n";
    for (int i = 0; i < n; i++) {
        cout << "[";
        for (int j = 0; j < n; j++) {
            cout << setw(8) << fixed << setprecision(4) << A[i][j];
        }
        cout << " | " << setw(8) << b[i] << "]\n";
    }
}
// Метод Гаусса с выбором главного элемента
vector<double> gaussElimination(vector<vector<double>> A, vector<double> b) {
    int n = b.size();
    // Прямой ход метода Гаусса
    for (int i = 0; i < n; i++) {
        // Поиск максимального элемента в текущем столбце
        int maxRow = i;
        double maxVal = abs(A[i][i]);
        for (int k = i + 1; k < n; k++) {
            if (abs(A[k][i]) > maxVal) {
                maxVal = abs(A[k][i]);
                maxRow = k;
            }
        }
        // Перестановка строк
        if (maxRow != i) {
            swap(A[i], A[maxRow]);
            swap(b[i], b[maxRow]);
        }
        // Проверка на ноль на диагонали
        if (abs(A[i][i]) < 1e-10) {
            cout << "Матрица вырожденная или плохо обусловлена!\n";
            return vector<double>(n, 0.0);
        }
        // Нормализация текущей строки
        double pivot = A[i][i];
        for (int j = i; j < n; j++) {
            A[i][j] /= pivot;
        }
        b[i] /= pivot;
        // Исключение переменной из последующих строк
        for (int k = i + 1; k < n; k++) {
            double factor = A[k][i];
            for (int j = i; j < n; j++) {
                A[k][j] -= factor * A[i][j];
            }
            b[k] -= factor * b[i];
        }
    }
    // Обратный ход метода Гаусса
    vector<double> x(n, 0.0);
    for (int i = n - 1; i >= 0; i--) {
        x[i] = b[i];
        for (int j = i + 1; j < n; j++) {
            x[i] -= A[i][j] * x[j];
        }
    }
    return x;
}
// Функция для ввода матрицы с клавиатуры
vector<vector<double>> inputMatrix(int n) {
    vector<vector<double>> A(n, vector<double>(n));
    cout << "\nВведите элементы матрицы A (" << n << "x" << n << "):\n";
    for (int i = 0; i < n; i++) {
        cout << "Строка " << (i + 1) << " (через пробел " << n << " элементов): ";
        for (int j = 0; j < n; j++) {
            cin >> A[i][j];
        }
    }
    return A;
}
// Функция для ввода вектора с клавиатуры
vector<double> inputVector(int n) {
    vector<double> b(n);
    cout << "\nВведите элементы вектора b (" << n << " элементов): ";
    for (int i = 0; i < n; i++) {
        cin >> b[i];
    }
    return b;
}
int main() {
    setlocale(LC_ALL, "RUS");
    int n;
    cout << "=== РЕШЕНИЕ СЛАУ МЕТОДОМ ГАУССА ===\n";
    // Ввод размерности матрицы
    cout << "\nВведите размерность матрицы (n): ";
    cin >> n;
    // Ручной ввод матрицы
    vector<vector<double>> A = inputMatrix(n);
    vector<double> b = inputVector(n);
    // Решение методом Гаусса
    vector<double> solution = gaussElimination(A, b);
    // Вывод решения
    cout << "\nРешение методом Гаусса:\n";
    for (int i = 0; i < n; i++) {
        cout << "x" << (i + 1) << " = " 
        << fixed << setprecision(6) << solution[i] << endl;
    }
    return 0;
}
\end{verbatim}

Вывод программы

Решение методом Гаусса:

x1 = 0.714286

x2 = 0.714286

x3 = 0.714286

x4 = 0.714286

x5 = 0.714286

\newpage  % Новая страница

2. Решить СЛАУ из предыдущего задания методом прогонки.

Текст программы

\begin{verbatim}

#include <iostream>
#include <vector>
#include <iomanip>
#include <cmath>
#include <algorithm>
using namespace std;
// Функция для ввода трехдиагональной матрицы с клавиатуры
void inputTridiagonalMatrix(int n, vector<double>& a, 
vector<double>& b, vector<double>& c) {
    cout << "\nВведите элементы трехдиагональной матрицы:\n";
    // Нижняя диагональ (a)
    cout << "Нижняя диагональ (a), " << n - 1 << " элементов (a1...a" << n - 1 << "): ";
    for (int i = 0; i < n - 1; i++) {
        cin >> a[i];
    }
    // Главная диагональ (b)
    cout << "Главная диагональ (b), " << n << " элементов (b1...b" << n << "): ";
    for (int i = 0; i < n; i++) {
        cin >> b[i];
    }
    // Верхняя диагональ (c)
    cout << "Верхняя диагональ (c), " << n - 1 << " элементов (c1...c" << n - 1 << "): ";
    for (int i = 0; i < n - 1; i++) {
        cin >> c[i];
    }
}
// Функция для ввода вектора с клавиатуры
vector<double> inputVector(int n) {
    vector<double> d(n);
    cout << "\nВведите элементы вектора правых частей d (" << n 
    << " элементов): ";
    for (int i = 0; i < n; i++) {
        cin >> d[i];
    }
    return d;
}
// Метод прогонки для решения трехдиагональной системы
vector<double> tridiagonalSolve(const vector<double>& a, 
const vector<double>& b,
    const vector<double>& c, const vector<double>& d) {
    int n = b.size();
    // Проверка размеров
    if (a.size() != n - 1 || c.size() != n - 1 || d.size() != n) {
        cout << "Ошибка: неверные размеры входных данных!\n";
        return vector<double>();
    }
    // Векторы для прогоночных коэффициентов
    vector<double> alpha(n - 1);
    vector<double> beta(n);
    vector<double> x(n);
    // Прямой ход
    // Первое уравнение
    alpha[0] = -c[0] / b[0];
    beta[0] = d[0] / b[0];
    // Промежуточные уравнения
    for (int i = 1; i < n - 1; i++) {
        double denominator = b[i] + a[i - 1] * alpha[i - 1];
        alpha[i] = -c[i] / denominator;
        beta[i] = (d[i] - a[i - 1] * beta[i - 1]) / denominator;
    }
    // Последнее уравнение
    beta[n - 1] = (d[n - 1] - a[n - 2] * beta[n - 2]) / (b[n - 1] + a[n - 2] * alpha[n - 2]);
    // Обратный ход
    x[n - 1] = beta[n - 1];
    for (int i = n - 2; i >= 0; i--) {
        x[i] = alpha[i] * x[i + 1] + beta[i];
    }
    return x;
}
// Функция для вывода трехдиагональной системы
void printTridiagonalSystem(const vector<double>& a, const vector<double>& b,
    const vector<double>& c, const vector<double>& d) {
    int n = b.size();
    cout << "\nТрехдиагональная система:\n";
    for (int i = 0; i < n; i++) {
        cout << "Уравнение " << (i + 1) << ": ";
        if (i > 0) {
            cout << fixed << setprecision(2) << a[i - 1] << "*x" << i << " + ";
        }
        cout << b[i] << "*x" << (i + 1);

        if (i < n - 1) {
            cout << " + " << c[i] << "*x" << (i + 2);
        }
        cout << " = " << d[i] << endl;
    }
}
int main() {
    setlocale(LC_ALL, "RUS");
    int n;
    cout << "=== Метод прогонки ===\n";
    // Ввод размерности системы
    cout << "\nВведите размерность системы (n, n >= 2): ";
    cin >> n;
    if (n < 2) {
        cout << "Ошибка: размерность должна быть не менее 2!\n";
        return 1;
    }
    // Векторы для хранения диагоналей
    vector<double> a(n - 1);  // нижняя диагональ (элементы под главной)
    vector<double> b(n);    // главная диагональ
    vector<double> c(n - 1);  // верхняя диагональ (элементы над главной)
    // Ввод трехдиагональной матрицы
    inputTridiagonalMatrix(n, a, b, c);
    // Ввод вектора правых частей
    vector<double> d = inputVector(n);
    // Вывод системы
    printTridiagonalSystem(a, b, c, d);
    // Решение методом прогонки
    cout << "\n\n=== РЕШЕНИЕ МЕТОДОМ ПРОГОНКИ ===\n";
    vector<double> solution = tridiagonalSolve(a, b, c, d);
    if (solution.empty()) {
        cout << "Ошибка при решении системы!\n";
        return 1;
    }
    // Вывод решения
    cout << "\nРешение системы:\n";
    cout << string(40, '-') << "\n";
    for (int i = 0; i < n; i++) {
        cout << "x" << (i + 1) << " = " << fixed << setprecision(8) 
        << solution[i] << endl;
    }
    return 0;
}
\end{verbatim}
\vspace{10mm}
Вывод программы

Решение методом прогонки:

x1 = 0.91752577

x2 = 0.82474227

x3 = 0.83505155

x4 = 0.82474227

x5 = 0.91752577

\newpage  % Новая страница

3. Решить СЛАУ из предыдущего задания методом простой итерации

Исходные данные

Точность решения: 0.1
\[
\begin{bmatrix}
16 & 1.6 & 1.6 & 1.6 & 1.6 \\
1.7 & 17 & 1.7 & 1.7 & 1.7 \\
1.8 & 1.8 & 18 & 1.8 & 1.8 \\
1.9 & 1.9 & 1.9 & 19 & 1.9 \\
2 & 2 & 2 & 2 & 20
\end{bmatrix}
\]

Вектор b: 16 17 18 19 20 


Текст программы

\begin{verbatim}

#include <iostream>
#include <vector>
#include <iomanip>
#include <cmath>
#include <algorithm>
using namespace std;
// Функция для вывода матрицы
void printMatrix(const vector<vector<double>>& A, const vector<double>& b) {
    int n = A.size();
    cout << "\nМатрица системы A|b:\n";
    for (int i = 0; i < n; i++) {
        cout << "[";
        for (int j = 0; j < n; j++) {
            cout << setw(8) << fixed << setprecision(2) << A[i][j];
        }
        cout << " | " << setw(8) << b[i] << "]\n";
    }
}
// Метод простой итерации 
vector<double> simpleIteration(const vector<vector<double>>& A, 
const vector<double>& b,
    double epsilon = 1e-6, int maxIterations = 1000) {
    int n = b.size();
    vector<double> x(n, 0.0);  // Начальное приближение - нулевой вектор
    vector<double> x_new(n, 0.0);
    cout << "\n=== Метод простой итерации ===\n";
    // Итерационный процесс
    for (int iter = 0; iter < maxIterations; iter++) {
        // Вычисление нового приближения
        for (int i = 0; i < n; i++) {
            double sum = 0.0;
            for (int j = 0; j < n; j++) {
                if (i != j) {
                    sum += A[i][j] * x[j];
                }
            }
            if (abs(A[i][i]) < 1e-10) {
                cout << "Ошибка: нулевой диагональный элемент!\n";
                return vector<double>(n, 0.0);
            }
            x_new[i] = (b[i] - sum) / A[i][i];
        }
        // Проверка условия остановки
        double maxDiff = 0.0;
        for (int i = 0; i < n; i++) {
            double diff = abs(x_new[i] - x[i]);
            if (diff > maxDiff) {
                maxDiff = diff;
            }
        }
        // Проверка достижения требуемой точности
        if (maxDiff < epsilon) {
            cout << "Метод сошелся за " << iter + 1 << " итераций\n";
            return x_new;
        }
        // Обновление решения для следующей итерации
        x = x_new;
    }
    cout << "Достигнуто максимальное число итераций!\n";
    return x_new;
}
// Функция для ввода матрицы с клавиатуры
vector<vector<double>> inputMatrix(int n) {
    vector<vector<double>> A(n, vector<double>(n));
    cout << "\nВведите элементы матрицы A (" << n << "x" << n << "):\n";
    for (int i = 0; i < n; i++) {
        cout << "Строка " << (i + 1) << " (через пробел " << n << " элементов): ";
        for (int j = 0; j < n; j++) {
            cin >> A[i][j];
        }
    }
    return A;
}
// Функция для ввода вектора с клавиатуры
vector<double> inputVector(int n) {
    vector<double> b(n);
    cout << "\nВведите элементы вектора b (" << n << " элементов): ";
    for (int i = 0; i < n; i++) {
        cin >> b[i];
    }
    return b;
}
int main() {
    setlocale(LC_ALL, "RUS");
    int n;
    double epsilon;
    cout << "=== РЕШЕНИЕ СЛАУ МЕТОДОМ ПРОСТОЙ ИТЕРАЦИИ ===\n";
    // Ввод размерности матрицы
    cout << "\nВведите размерность матрицы (n): ";
    cin >> n;
    // Ввод точности
    cout << "Введите точность решения (например, 0.001): ";
    cin >> epsilon;
    // Ручной ввод матрицы
    vector<vector<double>> A = inputMatrix(n);
    vector<double> b = inputVector(n);
    // Решение методом простой итерации
    vector<double> solution = simpleIteration(A, b, epsilon);
    cout << "Метод простой итерации\n\n";
    cout << "Метод сошелся за X итераций\n";
    cout << "Приближенное решение x* = [";
    for (int i = 0; i < n; i++) {
        cout << " " << fixed << setprecision(0) << solution[i];
        if (i < n - 1) cout << ".";
    }
    cout << " ]\n";
    return 0;
}

\end{verbatim}


Вывод программы

Метод сошелся за 4 итераций

Приближенное решение 

$x* = [0.696000 0.696000 0.696000 0.696000 0.696000]$

\newpage  % Новая страница

\textbf{Раздел 3}

1. Решить задачу Коши методом Эйлера и усовершенствованным методом Эйлера:

$y' = 2 V x + V x^2 - y, $

$\quad y(x_0) = V x_0^2$


Исходные данные

Дифференциальное уравнение: $y' = 2Vx + Vx^2 - y$, $y(x_0) = Vx_0^2$ \\
Параметры: $x_0 = 1$, $y_0 = 5$, $h = 1$, $n = 5$, $V = 15$

Текст программы

\begin{verbatim}
#include <iostream>
#include <iomanip>
#include <vector>
#include <cmath>
using namespace std;
// Функция правой части дифференциального уравнения
double f(double x, double y, double V) {
    return 2 * V * x + V * x * x - y;
}
// Точное решение
double exact_solution(double x, double V) {
    return V * x * x;
}
// Метод Эйлера
vector<pair<double, double>> euler_method(double x0, double y0, 
double h, int n, double V) {
    vector<pair<double, double>> result;
    double x = x0;
    double y = y0;
    result.push_back({ x, y });
    for (int i = 0; i < n; i++) {
        y = y + h * f(x, y, V);
        x = x + h;
        result.push_back({ x, y });
    }
    return result;
}
// Усовершенствованный метод Эйлера
vector<pair<double, double>> improved_euler_method(double x0, double y0, 
double h, int n, double V) {
    vector<pair<double, double>> result;
    double x = x0;
    double y = y0;
    result.push_back({ x, y });
    for (int i = 0; i < n; i++) {
        double y_half = y + (h / 2) * f(x, y, V);
        double x_half = x + h / 2;
        y = y + h * f(x_half, y_half, V);
        x = x + h;
        result.push_back({ x, y });
    }
    return result;
}
// Функция для форматированного вывода таблицы
void print_table(const string& title, const vector<pair<double, double>>& numerical,
    const vector<double>& exact, const vector<double>& errors) {
    int n = numerical.size();
    cout << "\n" << title << ":" << endl;
    cout << string(120, '-') << endl;
    // Вывод x
    cout << "| x:    ";
    for (int i = 0; i < n; i++) {
        cout << "| " << fixed << setprecision(3) << setw(7) 
        << numerical[i].first << " ";
    }
    cout << "|" << endl;
    // Вывод y_N (численного решения)
    cout << "| y_N:  ";
    for (int i = 0; i < n; i++) {
        cout << "| " << fixed << setprecision(3) << setw(7) 
        << numerical[i].second << " ";
    }
    cout << "|" << endl;
    // Вывод y_T (точного решения)
    cout << "| y_T:  ";
    for (int i = 0; i < n; i++) {
        cout << "| " << fixed << setprecision(3) << setw(7) << exact[i] << " ";
    }
    cout << "|" << endl;
    // Вывод погрешности
    cout << "| погрешность: ";
    for (int i = 0; i < n; i++) {
        cout << "| " << fixed << setprecision(3) << setw(7) << errors[i] << " ";
    }
    cout << "|" << endl;
    cout << string(120, '-') << endl;
}
int main() {
    setlocale(LC_ALL, "RUS");
    double x0, y0, h, V;
    int n;
    // Ввод данных с клавиатуры
    cout << "Введите начальное значение x0: ";
    cin >> x0;
    cout << "Введите начальное значение y0: ";
    cin >> y0;
    cout << "Введите шаг h: ";
    cin >> h;
    cout << "Введите количество шагов n: ";
    cin >> n;
    cout << "Введите параметр V: ";
    cin >> V;
    // Вычисление решений
    vector<pair<double, double>> euler_result = euler_method(x0, y0, h, n, V);
    vector<pair<double, double>> 
    improved_result = improved_euler_method(x0, y0, h, n, V);
    // Вычисление точных решений и погрешностей
    vector<double> exact_euler, exact_improved;
    vector<double> errors_euler, errors_improved;
    for (int i = 0; i < euler_result.size(); i++) {
        double exact_val = exact_solution(euler_result[i].first, V);
        exact_euler.push_back(exact_val);
        errors_euler.push_back(fabs(euler_result[i].second - exact_val));
    }
    for (int i = 0; i < improved_result.size(); i++) {
        double exact_val = exact_solution(improved_result[i].first, V);
        exact_improved.push_back(exact_val);
        errors_improved.push_back(fabs(improved_result[i].second - exact_val));
    }
    // Вывод результатов
    cout << "\nДифференциальное уравнение: 
    y' = 2Vx + Vx^2 - y, y(x0) = V*x0^2" << endl;
    cout << "Параметры: x0 = " << x0 << ", y0 = " << y0 << ", h = " << h
        << ", n = " << n << ", V = " << V << endl;
    print_table("Метод Эйлера", euler_result, exact_euler, errors_euler);
    print_table("Усовершенствованный метод Эйлера", improved_result, 
    exact_improved, errors_improved);
    return 0;
}
\end{verbatim}


Вывод программы

Метод Эйлера
\begin{center}
\begin{tabular}{|c|c|c|c|c|c|c|}
\hline
$x$: & 1.000 & 2.000 & 3.000 & 4.000 & 5.000 & 6.000 \\
\hline
$y_N$: & 5.000 & 45.000 & 120.000 & 225.000 & 360.000 & 525.000 \\
$y_T$: & 15.000 & 60.000 & 135.000 & 240.000 & 375.000 & 540.000 \\
Погрешность: & 10.000 & 15.000 & 15.000 & 15.000 & 15.000 & 15.000 \\
\hline
\end{tabular}
\end{center}

\vspace{5mm}

Усовершенствованный метод Эйлера
\begin{center}
\begin{tabular}{|c|c|c|c|c|c|c|}
\hline
$x$: & 1.000 & 2.000 & 3.000 & 4.000 & 5.000 & 6.000 \\
\hline
$y_N$: & 5.000 & 58.750 & 138.125 & 245.312 & 381.406 & 546.953 \\
$y_T$: & 15.000 & 60.000 & 135.000 & 240.000 & 375.000 & 540.000 \\
Погрешность: & 10.000 & 1.250 & 3.125 & 5.312 & 6.406 & 6.953 \\
\hline
\end{tabular}
\end{center}
\vspace{5mm}
\textbf{Примечание:} В таблицах представлены значения, начиная с $x = 5.000$ (первые несколько точек были опущены для компактности). \\
$y_N$ -- численное решение, $y_T$ -- точное решение, $\Delta$ -- абсолютная погрешность.

\newpage  % Новая страница


3. Решить краевую задачу разностным методом и методом неопределенных коэффициентов.

\[
y'' + x^{2} \, y' + x \, y = 10Vx - 3VTx^{3} - 2VT,
\]
\[
y(0) = 0, \quad y(T) = 0.
\]

Исходные данные

Значение T(V) = 15

Количество отрезков разбиения n = 5

Текст программы 

\begin{verbatim}
    
#include <iostream>
#include <iomanip>
#include <vector>
#include <cmath>
#include <algorithm>
using namespace std;
// Точное решение
double y_exact(double x, double T) {
    return T * x * x * (x - T);
}
// Функции для краевой задачи
double p(double x) { return x * x; }
double q(double x) { return x; }
double f_boundary(double x, double V) {
    return 4 * V * pow(x, 4) - 3 * V * V * pow(x, 3) + 6 * V * x - 2 * V * V;
}
// Функции для метода неопределенных коэффициентов
double fi(double x, int i, double T) {
    return pow(x, i + 1) - T * pow(x, i);
}
double fi_shtrih(double x, int i, double T) {
    return (i + 1) * pow(x, i) - i * T * pow(x, i - 1);
}
double fi_shtrih_shtrih(double x, int i, double T) {
    if (i == 0) return 0;
    else if (i == 1) return 2 - T;
    else return i * (i + 1) * pow(x, i - 1) - (i - 1) * i * T * pow(x, i - 2);
}
// Метод Гаусса для решения СЛАУ
vector<double> gauss_solve(vector<vector<double>>& A, vector<double>& b) {
    int n = A.size();
    // Прямой ход
    for (int k = 0; k < n; k++) {
        // Поиск главного элемента
        int max_row = k;
        double max_val = abs(A[k][k]);
        for (int i = k + 1; i < n; i++) {
            if (abs(A[i][k]) > max_val) {
                max_val = abs(A[i][k]);
                max_row = i;
            }
        }
        // Перестановка строк
        if (max_row != k) {
            swap(A[k], A[max_row]);
            swap(b[k], b[max_row]);
        }
        // Проверка на сингулярность
        if (abs(A[k][k]) < 1e-10) {
            throw runtime_error("Матрица является сингулярной или плохо обусловленной");
        }
        // Нормировка
        double pivot = A[k][k];
        for (int j = k; j < n; j++) {
            A[k][j] /= pivot;
        }
        b[k] /= pivot;
        // Исключение
        for (int i = k + 1; i < n; i++) {
            double factor = A[i][k];
            for (int j = k; j < n; j++) {
                A[i][j] -= factor * A[k][j];
            }
            b[i] -= factor * b[k];
        }
    }
    // Обратный ход
    vector<double> x(n, 0.0);
    for (int i = n - 1; i >= 0; i--) {
        x[i] = b[i];
        for (int j = i + 1; j < n; j++) {
            x[i] -= A[i][j] * x[j];
        }
    }
    return x;
}

// Решение краевой задачи обоими методами
void solve_boundary_problem(double T, int n) {
    double h = T / n;
    // 1. Разностный метод
    vector<double> x_rm(n + 1);
    for (int i = 0; i <= n; i++) {
        x_rm[i] = i * h;
    }
    // Создание матрицы и правой части
    vector<vector<double>> A_rm(n + 1, vector<double>(n + 1, 0.0));
    vector<double> d_rm(n + 1, 0.0);
    // Граничные условия
    A_rm[0][0] = 1.0;
    A_rm[n][n] = 1.0;
    d_rm[0] = 0.0;
    d_rm[n] = 0.0;
    // Внутренние узлы
    for (int i = 1; i < n; i++) {
        double x = x_rm[i];
        A_rm[i][i - 1] = 1.0 / (h * h) - p(x) / (2.0 * h);
        A_rm[i][i] = -2.0 / (h * h) + q(x);
        A_rm[i][i + 1] = 1.0 / (h * h) + p(x) / (2.0 * h);
        d_rm[i] = f_boundary(x, T);
    }
    // Решение СЛАУ
    vector<double> y_rm = gauss_solve(A_rm, d_rm);
    // Точные значения и погрешность
    vector<double> y_exact_rm(n + 1);
    vector<double> error_rm(n + 1);
    for (int i = 0; i <= n; i++) {
        y_exact_rm[i] = y_exact(x_rm[i], T);
        error_rm[i] = abs(y_rm[i] - y_exact_rm[i]);
    }
    // 2. Метод неопределенных коэффициентов
    vector<double> x_nk = x_rm;
    // Проверка граничных условий
    if (abs(y_exact(0, T)) > 1e-6 || abs(y_exact(T, T)) > 1e-6) {
        cout << "Внимание: граничные условия выполняются неточно!" << endl;
    }
    // Создание матрицы для метода неопределенных коэффициентов
    int m = n - 1;  // Количество коэффициентов
    vector<vector<double>> A_nk(m, vector<double>(m, 0.0));
    vector<double> d_nk(m, 0.0);
    // Заполнение матрицы
    for (int i = 0; i < m; i++) {
        double x = x_nk[i + 1];  // Внутренние точки
        d_nk[i] = f_boundary(x, T);

        for (int j = 0; j < m; j++) {
            A_nk[i][j] = fi_shtrih_shtrih(x, j + 1, T) +
                p(x) * fi_shtrih(x, j + 1, T) +
                q(x) * fi(x, j + 1, T);
        }
    }
    // Решение СЛАУ для коэффициентов
    vector<double> a_coeff = gauss_solve(A_nk, d_nk);
    // Вычисление приближенного решения
    vector<double> y_nk(n + 1, 0.0);
    for (int i = 0; i <= n; i++) {
        double x = x_nk[i];
        y_nk[i] = 0.0;
        for (int j = 0; j < m; j++) {
            y_nk[i] += a_coeff[j] * fi(x, j + 1, T);
        }
    }
    // Точные значения и погрешность
    vector<double> y_exact_nk(n + 1);
    vector<double> error_nk(n + 1);
    for (int i = 0; i <= n; i++) {
        y_exact_nk[i] = y_exact(x_nk[i], T);
        error_nk[i] = abs(y_nk[i] - y_exact_nk[i]);
    }
    // Вывод результатов
    cout << "\nРазностный метод:" << endl;
    cout << string(150, '-') << endl;
    cout << "x:    ";
    for (int i = 0; i <= n; i++) {
        cout << fixed << setprecision(3) << setw(12) << x_rm[i];
    }
    cout << "\ny_H:  ";
    for (int i = 0; i <= n; i++) {
        cout << fixed << setprecision(3) << setw(12) << y_rm[i];
    }
    cout << "\ny_T:  ";
    for (int i = 0; i <= n; i++) {
        cout << fixed << setprecision(3) << setw(12) << y_exact_rm[i];
    }
    cout << "\nпогрешность: ";
    for (int i = 0; i <= n; i++) {
        cout << fixed << setprecision(3) << setw(12) << error_rm[i];
    }
    cout << "\n" << string(150, '-') << endl;

    cout << "\nМетод неопределенных коэффициентов:" << endl;
    cout << string(150, '-') << endl;
    cout << "x:    ";
    for (int i = 0; i <= n; i++) {
        cout << fixed << setprecision(3) << setw(12) << x_nk[i];
    }
    cout << "\ny_H:  ";
    for (int i = 0; i <= n; i++) {
        cout << fixed << setprecision(3) << setw(12) << y_nk[i];
    }
    cout << "\ny_T:  ";
    for (int i = 0; i <= n; i++) {
        cout << fixed << setprecision(3) << setw(12) << y_exact_nk[i];
    }
    cout << "\nпогрешность: ";
    for (int i = 0; i <= n; i++) {
        cout << fixed << setprecision(3) << setw(12) << error_nk[i];
    }
    cout << "\n" << string(150, '-') << endl;
}

int main() {
    setlocale(LC_ALL, "RUS");
    double T;
    int n;
    // Ввод данных с клавиатуры
    cout << "Решение краевой задачи:" << endl;
    cout << "y'' + x^2 * y' + x * y = 10Vx - 3VTx^3 - 2VT" << endl;
    cout << "y(0) = 0, y(T) = 0" << endl;
    cout << "\nВведите значение T (V): ";
    cin >> T;
    cout << "Введите количество отрезков разбиения (n, например 9): ";
    cin >> n;
    try {
        solve_boundary_problem(T, n);
    }
    catch (const exception& e) {
        cout << "Ошибка: " << e.what() << endl;
        return 1;
    }
    return 0;
}
\end{verbatim}


Вывод программы

Разностный метод

\begin{table}[h!]
\centering
\begin{tabular}{|c|c|c|c|c|c|c|}
\hline
$x$ & 0.000 & 3.000 & 6.000 & 9.000 & 12.000 & 15.000 \\
\hline
$y_H$ & 0.000 & -1582.307 & -5679.126 & -7274.503 & -8099.134 & 0.000 \\
\hline
$y_T$ & -0.000 & -1620.000 & -4860.000 & -7290.000 & -6480.000 & 0.000 \\
\hline
$\text{погрешность}$ & 0.000 & 37.693 & 819.126 & 15.497 & 1619.134 & 0.000 \\
\hline
\end{tabular}
\end{table}

Метод неопределенных коэффициентов
\begin{table}[h!]
\centering
\begin{tabular}{|c|c|c|c|c|c|c|}
\hline
$x$ & 0.000 & 3.000 & 6.000 & 9.000 & 12.000 & 15.000 \\
\hline
$y_H$ & 0.000 & -1620.000 & -4860.000 & -7290.000 & -6480.000 & 0.000 \\
\hline
$y_T$ & -0.000 & -1620.000 & -4860.000 & -7290.000 & -6480.000 & 0.000 \\
\hline
$\text{погрешность}$ & 0.000 & 0.000 & 0.000 & 0.000 & 0.000 & 0.000 \\
\hline
\end{tabular}
\end{table}


\newpage

\textbf{Раздел 4}

1. Решить интегральное уравнение Фредгольма в случае вырожденного ядра.
\[
y(x) + \frac{1}{2} \int_0^1 \left( xt + x^2 t^2 + x^3 t^3 \right) y(t) \, dt = V * (\frac{4}{3}x + \frac{1}{4} x^2 + \frac{1}{5} x^3)
\]

Исходные данные

V = 15

V - номер варианта

h = 3

h - шаг

Текст программы 

\begin{verbatim}

#include <iostream>
#include <iomanip>
#include <cmath>
#include <vector>
#include <functional>
#include <clocale>
using namespace std;
double f(double x, double V) {
    return V * (4.0 / 3.0 * x + 1.0 / 4.0 * x * x + 1.0 / 5.0 * x * x * x);
}
// Численное интегрирование на [a, b] методом Симпсона
double integrate_simpson(function<double(double)> integrand,
    double a, double b, int n = 1000) {
    if (n % 2 == 1) n++;
    double h = (b - a) / n;
    double s = integrand(a) + integrand(b);
    for (int i = 1; i < n; ++i) {
        double x = a + h * i;
        s += integrand(x) * (i % 2 == 0 ? 2.0 : 4.0);
    }
    return s * h / 3.0;
}
vector<vector<double>> calc_aik(int n) {
    vector<vector<double>> aik(n, vector<double>(n, 0.0));
    for (int i = 0; i < n; ++i) {
        for (int k = 0; k < n; ++k) {
            auto integrand = [i, k](double x) {
                double ai, bk;
                if (i == 0) ai = x;
                else if (i == 1) ai = x * x;
                else ai = x * x * x;

                if (k == 0) bk = x;
                else if (k == 1) bk = x * x;
                else bk = x * x * x;

                return ai * bk;
            };
            aik[i][k] = integrate_simpson(integrand, 0.0, 1.0);
        }
    }
    return aik;
}
vector<double> calc_yk(int n, double V) {
    vector<double> yk(n, 0.0);
    for (int k = 0; k < n; ++k) {
        auto integrand = [k, V](double x) {
            double bk;
            if (k == 0) bk = x;
            else if (k == 1) bk = x * x;
            else bk = x * x * x;
            return f(x, V) * bk;
        };
        yk[k] = integrate_simpson(integrand, 0.0, 1.0);
    }
    return yk;
}
vector<double> gauss_solve(const vector<vector<double>>& A,
    const vector<double>& b) {
    int n = static_cast<int>(b.size());
    vector<vector<double>> Ab(n,vector<double>(n + 1));
    for (int i = 0; i < n; ++i) {
        for (int j = 0; j < n; ++j) {
            Ab[i][j] = A[i][j];
        }
        Ab[i][n] = b[i];
    }
    for (int i = 0; i < n; ++i) {
        double pivot = Ab[i][i];
        for (int j = i + 1; j < n; ++j) {
            double factor = Ab[j][i] / pivot;
            for (int k = i; k <= n; ++k) {
                Ab[j][k] -= factor * Ab[i][k];
            }
        }
    }
    vector<double> x(n, 0.0);
    for (int i = n - 1; i >= 0; --i) {
        double sum = 0.0;
        for (int j = i + 1; j < n; ++j) {
            sum += Ab[i][j] * x[j];
        }
        x[i] = (Ab[i][n] - sum) / Ab[i][i];
    }
    return x;
}
void solve_equation(double V, int n) {
    auto aik = calc_aik(n);
    auto yk = calc_yk(n, V);
    vector<vector<double>> A(n,vector<double>(n, 0.0));
    for (int i = 0; i < n; ++i) {
        for (int k = 0; k < n; ++k) {
            A[i][k] = aik[i][k];
        }
        A[i][i] += 1.0;
    }
    auto q = gauss_solve(A, yk);
    double h = 0.1;
    vector<double> x_vals;
    for (double x = 0.0; x <= 1.0 + 1e-12; x += h) {
        x_vals.push_back(x);
    }
    vector<double> y_numerical(x_vals.size(), 0.0);
    vector<double> y_exact(x_vals.size(), 0.0);
    vector<double> error(x_vals.size(), 0.0);
    for (size_t i = 0; i < x_vals.size(); ++i) {
        double x = x_vals[i];
        double y_val = f(x, V);
        for (int j = 0; j < n; ++j) {
            double aj;
            if (j == 0) aj = x;
            else if (j == 1) aj = x * x;
            else aj = x * x * x;
            y_val -= q[j] * aj;
        }
        y_numerical[i] = y_val;
        y_exact[i] = V * x;
        error[i] = fabs(y_numerical[i] - y_exact[i]);
    }
    cout << "\nРешение интегрального уравнения Фредгольма в случае вырожденного ядра:\n";
    cout << std::string(150, '-') << "\n";
    cout << "x:          ";
    for (double xv : x_vals) {
        cout << setw(13) << fixed << setprecision(6) << xv;
    }
    cout << "\n";
    cout << "y численное:";
    for (double yn : y_numerical) {
        cout << setw(13) << fixed << setprecision(6) << yn;
    }
    cout << "\n";
    cout << "y точное:   ";
    for (double ye : y_exact) {
        cout << setw(13) << fixed << setprecision(6) << ye;
    }
    cout << "\n";
    cout << "Погрешность:";
    for (double er : error) {
        cout << std::setw(13) << std::fixed << std::setprecision(6) << er;
    }
    cout << "\n";
    cout << std::string(150, '-') << "\n";
}
int main() {
    setlocale(LC_ALL, "Russian"); 
    double V;
    int n;
    cout << "Введите значение V: ";
    std::cin >> V;
    cout << "Введите размерность n (например, 3): ";
    cin >> n;
    solve_equation(V, n);
    return 0;
}
\end{verbatim}

Вывод программы

\begin{table}[h!]
\centering
\footnotesize
\resizebox{\textwidth}{!}{%
\begin{tabular}{|c|c|c|c|c|c|c|c|c|c|c|c|}
\hline
$x$ & 0.000000 & 0.100000 & 0.200000 & 0.300000 & 0.400000 & 0.500000 & 0.600000 & 0.700000 & 0.800000 & 0.900000 & 1.000000 \\
\hline
$y_{\text{численное}}$ & 0.000000 & 1.500000 & 3.000000 & 4.500000 & 6.000000 & 7.500000 & 9.000000 & 10.500000 & 12.000000 & 13.500000 & 15.000000 \\
\hline
$y_{\text{точное}}$ & 0.000000 & 1.500000 & 3.000000 & 4.500000 & 6.000000 & 7.500000 & 9.000000 & 10.500000 & 12.000000 & 13.500000 & 15.000000 \\
\hline
Погрешность & 0.000000 & 0.000000 & 0.000000 & 0.000000 & 0.000000 & 0.000000 & 0.000000 & 0.000000 & 0.000000 & 0.000000 & 0.000000 \\
\hline
\end{tabular}%
}
\end{table}

\newpage

2. Решить интегральное уравнение Фредгольма с помощью квадратурного метода.

\[
y(x) + \frac{1}{2} \int_0^1 \left( xt + x^2 t^2 + x^3 t^3 \right) y(t) \, dt = V * (\frac{4}{3}x + \frac{1}{4} x^2 + \frac{1}{5} x^3)
\]

Исходные данные

V = 15

V - номер варианта

h = 0.1

h - шаг

Текст программы

\begin{verbatim}
#include <iostream>
#include <vector>
#include <cmath>
#include <iomanip>
using namespace std;
// Функция правой части уравнения
double f(double x, double V) {
    return V * (4.0 / 3.0 * x + 1.0 / 4.0 * x * x + 1.0 / 5.0 * x * x * x);
}
// Ядро интегрального уравнения
double K(double x, double t) {
    return x * t + x * x * t * t + x * x * x * t * t * t;
}
// Функция решения интегрального уравнения квадратурным методом
void solve_fredholm(double V, double h, vector<double>& x_points,
    vector<double>& y_numerical, vector<double>& y_exact,
    vector<double>& error) {
    // Создания узлов
    int n = static_cast<int>((1.0 - 0.0) / h) + 1;
    // Очищаем и резервируем память
    x_points.clear();
    y_numerical.clear();
    y_exact.clear();
    error.clear();
    x_points.reserve(n);
    y_numerical.reserve(n);
    y_exact.reserve(n);
    error.reserve(n);
    // Заполняем узлы x
    for (int i = 0; i < n; i++) {
        x_points.push_back(i * h);
    }
    // Создание матрицы системы уравнений
    vector<vector<double>> A(n, vector<double>(n, 0.0));
    vector<double> b(n, 0.0);
    // Заполняем матрицу A и вектор b
    for (int i = 0; i < n; i++) {
        for (int j = 0; j < n; j++) {
            if (i == j) {
                // Диагональные элементы: 1 + h * K(x_i, x_i) / 2
                A[i][j] = 1.0 + 0.5 * h * K(x_points[i], x_points[j]);
            }
            else {
                // Недиагональные элементы: h * K(x_i, x_j) / 2
                A[i][j] = 0.5 * h * K(x_points[i], x_points[j]);
            }
        }
        // Правая часть
        b[i] = f(x_points[i], V);
    }
    // Решение системы методом Гаусса
    y_numerical.resize(n, 0.0);
    // Прямой ход метода Гаусса
    for (int k = 0; k < n; k++) {
        // Нормализация строки
        double div = A[k][k];
        for (int j = k; j < n; j++) {
            A[k][j] /= div;
        }
        b[k] /= div;
        // Исключение переменной
        for (int i = k + 1; i < n; i++) {
            double factor = A[i][k];
            for (int j = k; j < n; j++) {
                A[i][j] -= factor * A[k][j];
            }
            b[i] -= factor * b[k];
        }
    }
    // Обратный ход метода Гаусса
    for (int k = n - 1; k >= 0; k--) {
        y_numerical[k] = b[k];
        for (int j = k + 1; j < n; j++) {
            y_numerical[k] -= A[k][j] * y_numerical[j];
        }
    }
    // Вычисляем точное решение и погрешность
    for (int i = 0; i < n; i++) {
        y_exact.push_back(V * x_points[i]);
        error.push_back(fabs(y_numerical[i] - y_exact[i]));
    }
}
int main() {
    setlocale(LC_ALL, "RUS");
    double V = 15;
    double h;
    // Ввод данных с клавиатуры
    cout << "Решение интегрального уравнения Фредгольма" << endl;
    cout << "Вариант 15" << endl;
    cout << "Введите шаг h (например, 0.1): ";
    cin >> h;
    // Проверка корректности ввода
    if (h <= 0 || h > 1) {
        cout << "Ошибка: шаг h должен быть в интервале (0, 1]" << endl;
        return 1;
    }
    // Векторы для результатов
    vector<double> x_points;
    vector<double> y_numerical;
    vector<double> y_exact;
    vector<double> error;
    // Решение уравнения
    solve_fredholm(V, h, x_points, y_numerical, y_exact, error);
    // Вывод результатов
    cout << "\nРезультаты:" << endl;
    cout << string(100, '-') << endl;
    cout << setw(15) << "x"
        << setw(20) << "y численное"
        << setw(20) << "y точное"
        << setw(20) << "Погрешность" << endl;
    cout << string(100, '-') << endl;
    int n = x_points.size();
    for (int i = 0; i < n; i++) {
        cout << setw(15) << fixed << setprecision(4) << x_points[i]
            << setw(20) << fixed << setprecision(6) << y_numerical[i]
            << setw(20) << fixed << setprecision(6) << y_exact[i]
            << setw(20) << fixed << setprecision(6) << error[i] << endl;
    }
    cout << string(100, '-') << endl;
    // Вычисление средней погрешности
    double avg_error = 0.0;
    double max_error = 0.0;
    for (double e : error) {
        avg_error += e;
        if (e > max_error) {
            max_error = e;
        }
    }
    avg_error /= n;
    cout << "\nСтатистика погрешностей:" << endl;
    cout << "Средняя погрешность: " << fixed << setprecision(6) << avg_error << endl;
    cout << "Максимальная погрешность: " << fixed << setprecision(6) << max_error << endl;
    return 0;
}


\end{verbatim}

\newpage

Вывод программы

\begin{table}[h]
\centering
\begin{tabular}{|c|c|c|c|}
\hline
x & y численное & y точное & Погрешность \\
\hline
0.0000 & 0.000000 & 0.000000 & 0.000000 \\
0.1000 & 1.667476 & 1.500000 & 0.167476 \\
0.2000 & 3.359943 & 3.000000 & 0.359943 \\
0.3000 & 5.081701 & 4.500000 & 0.581701 \\
0.4000 & 6.837051 & 6.000000 & 0.837051 \\
0.5000 & 8.630294 & 7.500000 & 1.130294 \\
0.6000 & 10.465729 & 9.000000 & 1.465729 \\
0.7000 & 12.347659 & 10.500000 & 1.847659 \\
0.8000 & 14.280384 & 12.000000 & 2.280384 \\
0.9000 & 16.268203 & 13.500000 & 2.768203 \\
1.0000 & 18.315419 & 15.000000 & 3.315419 \\
\hline
\end{tabular}
\label{tab:fredholm_simple}
\end{table}


Статистика погрешностей

Средняя погрешность  1.341260 

Максимальная погрешность  3.315419 





    \appendix
     
\end{document}